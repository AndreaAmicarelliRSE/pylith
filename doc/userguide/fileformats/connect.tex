\subsection{xx.connect}

The \filename{xx.connect} file contains the finite-element mesh
topology and material type information, including the element type,
material type, and the lists of vertices for each element.

\begin{figure}
  \begin{center}
\begin{verbatim}
# File containing finite-element mesh topology and material type
# information.
#
# Comment lines begin with '#'
#
# Columns:
#   (1) Element number
#   (2) Element type
#        1 = Linear hexahedron (8 vertices)
#        2 = Linear hexahedron with 1 set of collapsed vertices
#            (7 vertices) [NOT IMPLEMENTED]
#        3 = Linear hexahedron with 2 sets of collapsed vertices
#           (6 vertices) [NOT IMPLEMENTED]
#        4 = Linear hexahedron with 4 vertices collapsed to a point
#            (5 vertices) [NOT IMPLEMENTED]
#        5 = Linear tetrahedron (4 vertices)
#        6 = Quadratix hexahedron (20 vertices) [NOT IMPLEMENTED]
#        7 = Quadratic hexahedron with 3 vertices along one edge 
#            collapsed to a point (18 vertices) [NOT IMPLEMENTED]
#        8 = Quadratic hexahedron with 3 sets of collapsed vertices
#            (15 vertices) [NOT IMPLEMENTED]
#        9 = Quadratic hexahedron with 9 vertices collapsed to a point
#            (13 vertices) [NOT IMPLEMENTED]
#       10 = Quadratic tetrahedron (10 vertices)  [NOT IMPLEMENTED]
#   (3) Material type, numbered consecutively beginning with '1'
#   (4) Infinite element flag, required but not currently implemented
#        0 = only valid value
#   (5)+ Vertices in the element, identified by vertex number
#
# Note: No comments are allowed within the mesh information.
#
1  5  1  0  2911  2865  2886  2864
2  5  2  0   843  3999  4029  3926
3  5  1  0   684  8975  2346  6219
\end{verbatim}
    \ldots
    \caption{Format of \filename{xx.connect} files.}
  \end{center}
\end{figure}  
