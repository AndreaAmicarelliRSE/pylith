\chapter{File Formats}

\section{Input Files}

PyLith gathers its input from several different types of files. All of
these area ASCII files and can include comment lines that begin with
'\#'. Note that the placement of comments is restricted to certain
locations in some files (see the discussion of each file format for
more information).

% REQUIRED INPUT FILES

\subsection{xx.coord}

The \filename{xx.coord} file contains the coordinates of all of the
vertices in the finite-element mesh.

\begin{figure}[htbp]
  \begin{center}
    \verbatiminput{data/xx.coord}
    \caption{Format of \filename{xx.coord} files.}
  \end{center}
\end{figure}

\subsection{xx.connect}

The \filename{xx.connect} file contains the finite-element mesh
topology and material type information, including the element type,
material type, and the lists of vertices for each element.

\begin{figure}[htbp]
  \begin{center}
    \verbatiminput{data/xx.connect}
    \caption{Format of \filename{xx.connect} files.}
  \end{center}
\end{figure}  

\subsection{xx.bc}

The \filename{xx.bc} file specifies the displacements, velocity,
and/or forces applied to vertices on the boundaries.

\begin{figure}[htbp]
  \begin{center}
    \verbatiminput{data/xx.bc}
    \caption{Format of \filename{xx.bc} files.}
  \end{center}
\end{figure}

\subsection{xx.time}

The \filename{xx.time} file specifies the time stepping parameters for
the simulation.

\begin{warning}
  The convergence criteria depend on the type of solution and material
  models. The time step for a linear elastic problem is much different
  than that for a nonlinear or time-dependent problem.
\end{warning}

\begin{figure}[htbp]
  \begin{center}
    \verbatiminput{data/xx.time}
    \caption{Format of \filename{xx.time} files.}
  \end{center}
\end{figure}

\subsection{xx.prop}

The \filename{xx.prop} file specifies the properties for each material
model in the problem.

\begin{warning}
  The materials must be listed in order according to the material
  number assigned to the elements in \filename{xx.connect}.
\end{warning}

\begin{figure}[htbp]
  \begin{center}
    \verbatiminput{data/xx.prop}
    \caption{Format of \filename{xx.prop} files.}
  \end{center}
\end{figure}

\subsection{xx.statevar}

The \filename{xx.statevar} file specifies which state variables are to
be included in the output of the elastic and time dependent solutions.

\begin{figure}[htbp]
  \begin{center}
    \verbatiminput{data/xx.statevar}
    \caption{Format of \filename{xx.statevar} files.}
  \end{center}
\end{figure}

% OPTIONAL INPUT FILES

\subsection{xx.split}

The \filename{xx.split} file specifies the split node information for
modeling dislocations. Dislocations may be used in simulating slip on
faults as well as dike intrusions.

\begin{figure}[htbp]
  \begin{center}
    \verbatiminput{xx.split}
    \caption{Format of \filename{xx.split} files.}
  \end{center}
\end{figure}

\subsection{xx.fuldat}

The \filename{xx.fuldat} file lists the time step numbers at which
full output is desired. The elastic solution (time step 0) is always
included in the output. This file is required for time-dependent
problems.

\begin{figure}[htbp]
  \begin{center}
    \verbatiminput{xx.fuldat}
    \caption{Format of \filename{xx.time} files.}
  \end{center}
\end{figure}

\subsection{xx.skew}

The \filename{xx.skew} file specifies local coordinate systems for
nodes. The local coordinate system is specified using two Euler angles
that rotate the local coordinate system to the global coordinate
system.

The applied coordinate rotations apply to all boundary conditions
associated with the nodes listed in the file. These are useful, for
example, if it is desired to apply boundary conditions in a direction
normal or tangential to a side of the mesh when the side does not
align with the global coordinate directions.  Similarly, skew
conditions could be used when specifying slip on a fault that lies at
an angle to the global coordinates.

\begin{figure}[htbp]
  \begin{center}
    \verbatiminput{xx.skew}
    \caption{Format of \filename{xx.time} files.}
  \end{center}
\end{figure}

\subsection{xx.keyval}

The \filename{xx.keyval} file specifies some simple parameter
settings.

\subsubsection{Winkler forces}

Scaling factors can be applied to Winkler forces, permitting a quick
and easy way to change the density or gravitational acceleration when
Winkler forces are used to simulate gravity.

\paragraph{Quadrature order}

\begin{description}
\item[Full] Quadrature order that should give the exact element
  matrices when the elements are geometrically undistorted.
\item[Reduced] Quadrature order that is one order less than full
  quadrature. Note that for linear tetrahedra full and reduced
  quadrature are equivalent (single integration point).

  \begin{warning}
    Use with caution as reduced quadrature can lead to numerical
    instabilities.
  \end{warning}
  
\item[Selective] Uses Hughes' b-bar formulation to perform reduced
  quadrature on the dilatational parts of the strain-displacement
  matrix.  This can be useful in nearly-incompressible problems.
\end{description}

\paragraph{Prestresses}

Gravitational prestresses can be computed automatically. In such
cases, the elastic properties in the prestress calculation can be set
to uniform values independent of the parameters for any of the
material models. When gravity is being used and prestresses are not
computed automatically, each prestress component can be scaled
independently. Reading prestresses from files is presently disabled.

\begin{figure}[htbp]
  \begin{center}
    \verbatiminput{xx.keyval}
  \caption{Format of \filename{xx.keyval} files.}
  \end{center}
\end{figure}

\subsection{xx.hist}

The \filename{xx.hist} files provide time histories for use in
boundary conditions.

\begin{figure}
  \begin{center}
    \verbatiminput{xx.hist}
    \caption{Format of \filename{xx.hist} files.}
  \end{center}
\end{figure}

\subsection{xx.wink}

The \filename{xx.wink} file specifies Winkler elements, which may be
used as spring foundations in the simulation of gravity.

\begin{figure}
  \begin{center}
    \verbatiminput{xx.wink}
    \caption{Format of \filename{xx.wink} files.}
  \end{center}
\end{figure}
