\subsection{xx.prop}

The \filename{xx.prop} file specifies the properties for each material
model in the problem.

\begin{warning}
  The materials must be listed in order according to the material
  number assigned to the elements in \filename{xx.connect}.
\end{warning}

\begin{figure}
  \begin{center}
\begin{verbatim}
# File containing material properties.
#
# Comment lines begin with '#'
#
# The material type and material property values are specified using a
# "keyword = value" syntax. The keywords for the different material
# types are given below. Units for each of the values with dimensions
# must follow the value as illustrated in the examples below.
#
# Materials and keywords:
#   Isotropic linear elastic
#     materialType ='IsotropicLinearElastic'
#     density
#     youngsModulus
#     poissonsRatio
#     endMaterial ='True' (flag indicating end of material)
#   Isotropic linear maxwell viscoelastic
#     materialType ='IsotropicLinearMaxwellViscoelastic'
#     density
#     youngsModulus
#     poissonsRatio
#     viscosity
#     endMaterial ='True' (flag indicating end of material)
#
# Material number 1
materialType = 'IsotropicLinearMaxwellViscoelastic'
density         = 3000.0*kg/m**3
youngsModulus   = 7.5e10*Pa
poissonsRatio   = 0.25
viscosity       = 1.0e+18*Pa*s
endMaterial     = True
# Material number 2
materialType = 'IsotropicLinearElastic'
density         = 3000.0*kg/m**3
youngsModulus   = 7.5e10*Pa
poissonsRatio   = 0.25
endMaterial     = True
# Material number 3
materialType = 'IsotropicLinearMaxwellViscoelastic'
density         = 3000.0*kg/m**3
youngsModulus   = 7.5e10*Pa
poissonsRatio   = 0.25
viscosity       = 1.0e+18*Pa*s
endMaterial     = True
\end{verbatim}
    \ldots
    \caption{Format of \filename{xx.prop} files.}
  \end{center}
\end{figure}
