
\chapter{\label{cha:Governing-Equations}Governing Equations}

We present here a brief derivation of the equations for both quasi-static
and dynamic computations. Since the general equations are the same
(except for the absence of inertial terms in the quasi-static case),
we first derive these equations. We then present solution methods
for each specific case. In all of our derivations, we use the notation
described in Table \ref{tab:Mathematical-notation} for both index
and vector notation. When using index notation, we use the common
convention that repeated indices indicate summation over the range
of the index.

\noindent \begin{center}
\begin{table}[H]
\caption{\label{tab:Mathematical-notation}Mathematical notation}


\noindent \centering{}\medskip{}
\begin{tabular}{|c|c|c|}
\hline 
\multicolumn{2}{|c|}{Symbol} & Description\tabularnewline
\cline{1-2} 
Index notation & Vector Notation & \tabularnewline
\hline 
\hline 
$a_{i}$ & $\overrightarrow{a}$ & Vector field a\tabularnewline
\hline 
$a_{ij}$ & $\underline{a}$ & Second order tensor field a\tabularnewline
\hline 
$u_{i}$ & $\overrightarrow{u}$ & Displacement vector field\tabularnewline
\hline 
$d_{i}$ & $\vec{{d}}$ & Fault slip vector field\tabularnewline
\hline 
$f_{i}$ & $\overrightarrow{f}$ & Body force vector field\tabularnewline
\hline 
$T_{i}$ & $\overrightarrow{T}$ & Traction vector field\tabularnewline
\hline 
$\sigma_{ij}$ & $\underline{\sigma}$ & Stress tensor field\tabularnewline
\hline 
$n_{i}$ & $\overrightarrow{n}$ & Normal vector field\tabularnewline
\hline 
$\rho$ & $\rho$ & Mass density scalar field\tabularnewline
\hline 
\end{tabular}
\end{table}

\par\end{center}


\section{Derivation of Elasticity Equation}


\subsection{Index Notation}

Consider volume $V$ bounded by surface $S$. Applying a Lagrangian
description of the conservation of momentum gives

\begin{equation}
\frac{\partial}{\partial t}\int_{V}\rho\frac{\partial u_{i}}{\partial t}\, dV=\int_{V}f_{i}\, dV+\int_{S}T_{i}\, dS.\label{eqn:momentum:index}
\end{equation}
The traction vector field is related to the stress tensor through
\begin{equation}
T_{i}=\sigma_{ij}n_{j},
\end{equation}
where $n_{j}$ is the vector normal to $S$. Substituting into equation
\ref{eqn:momentum:index} yields
\begin{equation}
\frac{\partial}{\partial t}\int_{V}\rho\frac{\partial u_{i}}{\partial t}\, dV=\int_{V}f_{i}\, dV+\int_{S}\sigma_{ij}n_{j}\, dS.
\end{equation}
Applying the divergence theorem,
\begin{equation}
\int_{V}a_{i,j}\: dV=\int_{S}a_{j}n_{j}\: dS,
\end{equation}
to the surface integral results in
\begin{equation}
\frac{\partial}{\partial t}\int_{V}\rho\frac{\partial u_{i}}{\partial t}\, dV=\int_{V}f_{i}\, dV+\int_{V}\sigma_{ij,j}\, dV,
\end{equation}
which we can rewrite as
\begin{equation}
\int_{V}\left(\rho\frac{\partial^{2}u_{i}}{\partial t^{2}}-f_{i}-\sigma_{ij,j}\right)\, dV=0.
\end{equation}
Because the volume $V$ is arbitrary, the integrand must be zero at
every location in the volume, so that we end up with
\begin{gather}
\rho\frac{\partial^{2}u_{i}}{\partial t^{2}}-f_{i}-\sigma_{ij,j}=0\text{ in }V,\\
\sigma_{ij}n_{j}=T_{i}\text{ on }S_{T}\text{,}\\
u_{i}=u_{i}^{o}\text{ on }S_{u}\text{, and}\\
R_{ki}(u_{i}^{+}-u_{i}^{-})=d_{k}\text{ on }S_{f}.
\end{gather}
We specify tractions, $T_{i}$, on surface $S_{f}$, displacements,
$u_{i}^{o}$, on surface $S_{u}$, and slip, $d_{k}$, on fault surface
$S_{f}$ (we will consider the case of fault constitutive models in
Section \ref{sec:fault}). The rotation matrix $R_{ki}$ transforms
vectors from the global coordinate system to the fault coordinate
system. Note that since both $T_{i}$ and $u_{i}$ are vector quantities,
there can be some spatial overlap of the surfaces $S_{T}$ and $S_{u}$;
however, the same degree of freedom cannot simultaneously have both
types of boundary conditions.


\subsection{Vector Notation}

Consider volume $V$ bounded by surface $S$. Applying a Lagrangian
description of the conservation of momentum gives

\begin{equation}
\frac{\partial}{\partial t}\int_{V}\rho\frac{\partial\vec{u}}{\partial t}\, dV=\int_{V}\overrightarrow{f}\, dV+\int_{S}\overrightarrow{T}\, dS.\label{eqn:momentum:vec}
\end{equation}
The traction vector field is related to the stress tensor through
\begin{equation}
\overrightarrow{T}=\underline{\sigma}\cdot\overrightarrow{n},
\end{equation}
where $\overrightarrow{n}$ is the vector normal to $S$. Substituting
into equation \ref{eqn:momentum:vec} yields
\begin{equation}
\frac{\partial}{\partial t}\int_{V}\rho\frac{\partial\overrightarrow{u}}{\partial t}\, dV=\int_{V}\overrightarrow{f}\, dV+\int_{S}\underline{\sigma}\cdot\overrightarrow{n}\, dS.
\end{equation}
Applying the divergence theorem,
\begin{equation}
\int_{V}\nabla\cdot\overrightarrow{a}\: dV=\int_{S}\overrightarrow{a}\cdot\overrightarrow{n}\: dS,
\end{equation}
to the surface integral results in
\begin{equation}
\frac{\partial}{\partial t}\int_{V}\rho\frac{\partial\overrightarrow{u}}{\partial t}\, dV=\int_{V}\overrightarrow{f}\, dV+\int_{V}\nabla\cdot\underline{\sigma}\, dV,
\end{equation}
which we can rewrite as
\begin{equation}
\int_{V}\left(\rho\frac{\partial^{2}\overrightarrow{u}}{\partial t^{2}}-\overrightarrow{f}-\nabla\cdot\overrightarrow{\sigma}\right)\, dV=\vec{0}.
\end{equation}
Because the volume $V$ is arbitrary, the integrand must be the zero
vector at every location in the volume, so that we end up with
\begin{gather}
\rho\frac{\partial^{2}\overrightarrow{u}}{\partial t^{2}}-\overrightarrow{f}-\nabla\cdot\overrightarrow{\sigma}=\vec{0}\text{ in }V,\\
\underline{\sigma}\cdot\overrightarrow{n}=\overrightarrow{T}\text{ on }S_{T}\text{,}\\
\overrightarrow{u}=\overrightarrow{u^{o}}\text{ on }S_{u},\text{ and}\\
\underbar{R}\cdot(\vec{u^{+}}-\vec{u^{-}})=\vec{d}\text{ on }S_{f}.
\end{gather}
We specify tractions, $\vec{T}$, on surface $S_{f}$, displacements,
$\overrightarrow{u^{o}}$, on surface $S_{u}$, and slip, $\vec{d}$,
on fault surface $S_{f}$ (we will consider the case of fault constitutive
models in Section \ref{sec:fault}). The rotation matrix $\underline{R}$
transforms vectors from the global coordinate system to the fault
coordinate system. Note that since both $\overrightarrow{T}$ and
$\overrightarrow{u}$ are vector quantities, there can be some spatial
overlap of the surfaces $S_{T}$ and $S_{u}$; however, the same degree
of freedom cannot simultaneously have both types of boundary conditions.


\section{Finite-Element Formulation of Elasticity Equation}

We formulate a set of algebraic equations using Galerkin's method.
We consider (1) a trial solution, $\vec{u}$, that is a piecewise
differentiable vector field and satisfies the Dirichlet boundary conditions
on $S_{u}$, and (2) a weighting function, $\vec{\phi}$, that is
a piecewise differentiable vector field and is zero on $S_{u}$.


\subsection{Index Notation}

We start with the wave equation (strong form),

\begin{gather}
\sigma_{ij,j}+f_{i}=\rho\ddot{u_{i}}\text{ in }V,\\
\sigma_{ij}n_{j}=T_{i}\text{ on }S_{T},\\
u_{i}=u_{i}^{o}\text{ on }S_{u},\\
R_{ki}(u_{i}^{+}-u_{i}^{-})=d_{k}\text{ on }S_{f},\text{ and}\\
\sigma_{ij}=\sigma_{ji}\text{ (symmetric).}
\end{gather}
We construct the weak form by computing the dot product of the wave
equation and weighting function and setting the integral over the
domain to zero:
\begin{gather}
\int_{V}\left(\sigma_{ij,j}+f_{i}-\rho\ddot{u}_{i}\right)\phi_{i}\, dV=0\text{, or }\\
\int_{V}\sigma_{ij,j}\phi_{i}\: dV+\int_{V}f_{i}\phi_{i}\: dV-\int_{V}\rho\ddot{u}_{i}\phi_{i}\: dV=0.
\end{gather}
 Consider the divergence theorem applied to the dot product of the
stress tensor and the weighting function, $\sigma_{ij}\phi_{i}$,
\begin{equation}
\int_{V}(\sigma_{ij}\phi_{i})_{,j}\, dV=\int_{S}(\sigma_{ij}\phi_{i})n_{i}\, dS.
\end{equation}
Expanding the left-hand side yields
\begin{gather}
\int_{V}\sigma_{ij,j}\phi_{i}\: dV+\int_{V}\sigma_{ij}\phi_{i,j}\: dV=\int_{S}\sigma_{ij}\phi_{i}n_{i}\: dS,\text{ or}\\
\int_{V}\sigma_{ij,j}\phi_{i}\: dV=-\int_{V}\sigma_{ij}\phi_{i,j}\, dV+\int_{S}\sigma_{ij}\phi_{i}n_{i}\, dS.
\end{gather}
Substituting into the weak form gives
\begin{equation}
-\int_{V}\sigma_{ij}\phi_{i,j}\, dV+\int_{S}\sigma_{ij}\phi_{i}n_{i}\, dS+\int_{V}f_{i}\phi_{i}\, dV-\int_{V}\rho\ddot{u}_{i}\phi_{i}\, dV=0.
\end{equation}
Turning our attention to the second term, we separate the integration
over $S$ into integration over $S_{T}$ and $S_{u}$ (we will consider
tractions over the fault surface, $S_{f}$, associated with the fault
constitutive model in Section \ref{sec:fault}),
\begin{equation}
-\int_{V}\sigma_{ij}\phi_{i,j}\, dV+\int_{S_{T}}\sigma_{ij}\phi_{i}n_{i}\, dS+\int_{S_{u}}\sigma_{ij}\phi_{i}n_{i}\, dS+\int_{V}f_{i}\phi_{i}\, dV-\int_{V}\rho\ddot{u}_{i}\phi_{i}\, dV=0,
\end{equation}
and recognize that
\begin{gather}
\sigma_{ij}n_{i}=T_{i}\text{ on }S_{T}\text{ and}\\
\phi_{i}=0\text{ on }S_{u},
\end{gather}
so that the equation reduces to
\begin{equation}
-\int_{V}\sigma_{ij}\phi_{i,j}\: dV+\int_{S_{T}}T_{i}\phi_{i}\, dS+\int_{V}f_{i}\phi_{i}\, dV-\int_{V}\rho\ddot{u}_{i}\phi_{i}\, dV=0.\label{eq:elasticity:integral}
\end{equation}
We express the trial solution and weighting function as linear combinations
of basis functions,
\begin{gather}
u_{i}=\sum_{m}a_{i}^{m}N^{m},\\
\phi_{i}=\sum_{n}c_{i}^{n}N^{n}.
\end{gather}
Note that because the trial solution satisfies the Dirichlet boundary
condition, the number of basis functions for $u$ is generally greater
than the number of basis functions for $\phi$, i.e., $m>n$. Substituting
in the expressions for the trial solution and weighting function yields
\begin{gather}
-\int_{V}\sigma_{ij}\sum_{n}c_{i}^{n}N_{,j}^{n}\: dV+\int_{S_{T}}T_{i}\sum_{n}c_{i}^{n}N^{n}\, dS+\int_{V}f_{i}\sum_{n}c_{i}^{n}N^{n}\, dV-\int_{V}\rho\sum_{m}\ddot{a}_{i}^{m}N^{m}\sum_{n}c_{i}^{n}N^{n}\ dV=0,\text{ or}\\
\sum_{n}c_{i}^{n}(-\int_{V}\sigma_{ij}N_{,j}^{n}\: dV+\int_{S_{T}}T_{i}N^{n}\, dS+\int_{V}f_{i}N^{n}\, dV-\int_{V}\rho\sum_{m}\ddot{a}_{i}^{m}N^{m}N^{n}\ dV)=0.
\end{gather}
 Because the weighting function is arbitrary, this equation must hold
for all $c_{i}^{n}$, so that the quantity in parenthesis is zero
for each $c_{i}^{n}$
\begin{equation}
-\int_{V}\sigma_{ij}N_{,j}^{n}\: dV+\int_{S_{T}}T_{i}N^{n}\, dS+\int_{V}f_{i}N^{n}\, dV-\int_{V}\rho\sum_{m}\ddot{a}_{i}^{m}N^{m}N^{n}\ dV=\vec{0}.\label{eq:elasticity:integral:discretized}
\end{equation}
We want to solve this equation for the unknown coefficients $a_{i}^{m}$
subject to

\begin{gather}
u_{i}=u_{i}^{o}\text{ on }S_{u},\text{ and}\\
R_{ki}(u_{i}^{+}-u_{i}^{-})=d_{k}\text{ on }S_{f},
\end{gather}



\subsection{Vector Notation}

We start with the wave equation (strong form),

\begin{gather}
\nabla\cdot\underline{\sigma}+\overrightarrow{f}=\rho\frac{\partial^{2}\overrightarrow{u}}{\partial t^{2}}\text{ in }V,\\
\underline{\sigma}\cdot\overrightarrow{n}=\overrightarrow{T}\text{ on }S_{T},\\
\overrightarrow{u}=\overrightarrow{u^{o}}\text{ on }S_{u},\\
\underbar{R}\cdot(\overrightarrow{u^{+}}-\overrightarrow{u^{-}})=\vec{d}\text{ on }S_{f}\\
\underline{\sigma}=\underline{\sigma}^{T}\text{ (symmetric).}
\end{gather}
We construct the weak form by multiplying the wave equation by a weighting
function and setting the integral over the domain to zero. The weighting
function is a piecewise differential vector field, $\overrightarrow{\phi}$,
where $\overrightarrow{\phi}=0$ on $S_{u}.$ Hence our weak form
is
\begin{gather}
\int_{V}\left(\nabla\cdot\underline{\sigma}+\overrightarrow{f}-\rho\frac{\partial^{2}\overrightarrow{u}}{\partial t^{2}}\right)\cdot\overrightarrow{\phi}\, dV=0\text{, or }\\
\int_{V}(\nabla\cdot\underline{\sigma})\cdot\overrightarrow{\phi}\: dV+\int_{V}\overrightarrow{f}\cdot\overrightarrow{\phi}\: dV-\int_{V}\rho\frac{\partial^{2}\overrightarrow{u}}{\partial t^{2}}\cdot\overrightarrow{\phi}\: dV=0.
\end{gather}
 Consider the divergence theorem applied to the dot product of the
stress tensor and the trial function, $\underline{\sigma}\cdot\overrightarrow{\phi}$,
\begin{equation}
\int_{V}\nabla\cdot(\underline{\sigma}\cdot\overrightarrow{\phi})\, dV=\int_{S}(\underline{\sigma}\cdot\overrightarrow{\phi})\cdot\overrightarrow{n}\, dS.
\end{equation}
Expanding the left-hand side yields
\begin{equation}
\int_{V}(\nabla\cdot\underline{\sigma})\cdot\overrightarrow{\phi}\: dV+\int_{V}\underline{\sigma}:\nabla\overrightarrow{\phi}\: dV=\int_{S}(\underline{\sigma}\cdot\overrightarrow{\phi})\cdot\overrightarrow{n}\: dS,\text{ or}
\end{equation}
\begin{equation}
\int_{V}(\nabla\cdot\underline{\sigma})\cdot\overrightarrow{\phi}\: dV=-\int_{V}\underline{\sigma}:\nabla\overrightarrow{\phi}\, dV+\int_{S}\underline{\sigma}\cdot\overrightarrow{n}\cdot\overrightarrow{\phi}\, dS.
\end{equation}
Substituting into the weak form gives
\begin{equation}
-\int_{V}\underline{\sigma}:\nabla\overrightarrow{\phi}\, dV+\int_{S}\underline{\sigma}\cdot\overrightarrow{n}\cdot\overrightarrow{\phi}\, dS+\int_{V}\overrightarrow{f}\cdot\overrightarrow{\phi}\, dV-\int_{V}\rho\frac{\partial^{2}\overrightarrow{u}}{\partial t^{2}}\cdot\overrightarrow{\phi}\, dV=0.
\end{equation}
We separate the integration over $S$ into integration over $S_{T}$
and $S_{u}$,
\begin{multline}
-\int_{V}\underline{\sigma}:\nabla\overrightarrow{\phi}\, dV+\int_{S_{T}}\underline{\sigma}\cdot\overrightarrow{n}\cdot\overrightarrow{\phi}\, dS+\int_{S_{u}}\underline{\sigma}\cdot\overrightarrow{n}\cdot\overrightarrow{\phi}\, dS+\int_{V}\overrightarrow{f}\cdot\overrightarrow{\phi}\, dV-\int_{V}\rho\frac{\partial^{2}\overrightarrow{u}}{\partial t^{2}}\cdot\overrightarrow{\phi}\, dV=0,
\end{multline}
and recognize that
\begin{gather}
\underline{\sigma}\cdot\overrightarrow{n}=\overrightarrow{T}\text{ on }S_{T}\text{ and}\\
\overrightarrow{\phi}=0\text{ on }S_{u},
\end{gather}
so that the equation reduces to
\begin{equation}
-\int_{V}\underline{\sigma}:\nabla\overrightarrow{\phi}\: dV+\int_{S_{T}}\overrightarrow{T}\cdot\overrightarrow{\phi}\, dS+\int_{V}\overrightarrow{f}\cdot\overrightarrow{\phi}\, dV-\int_{V}\rho\frac{\partial^{2}\overrightarrow{u}}{\partial t^{2}}\cdot\overrightarrow{\phi}\, dV=0.
\end{equation}
We express the trial solution and weighting function as linear combinations
of basis functions,
\begin{gather}
\vec{u}=\sum_{m}\overrightarrow{a^{m}}N^{m},\\
\vec{\phi}=\sum_{n}\overrightarrow{c^{n}}N^{n}.
\end{gather}
Note that because the weighting function is zero on $S_{u}$, the
number of basis functions for $\vec{u}$ is generally greater than
the number of basis functions for $\vec{\phi}$, i.e., $m>n$. Substituting
in the expressions for the trial solution and weighting function yields
\begin{multline}
-\int_{V}\underline{\sigma}:\sum_{n}\overrightarrow{c^{n}}\nabla N_{,}^{n}\, dV+\int_{S_{T}}\vec{T}\cdot\sum_{n}\overrightarrow{c^{n}}N^{n}\, dS+\int_{V}\vec{f}\cdot\sum_{n}\overrightarrow{c^{n}}N^{n}\, dV\\
-\int_{V}\rho\sum_{m}\frac{\partial^{2}\overrightarrow{a^{m}}}{\partial t^{2}}N^{m}\cdot\sum_{n}\overrightarrow{c^{n}}N^{n}\ dV=0.
\end{multline}
 Because the weighting function is arbitrary, this equation must hold
for all $\overrightarrow{c^{n}}$, so that
\begin{equation}
-\int_{V}\underline{\sigma}:\nabla N^{n}\, dV+\int_{S_{T}}\vec{T}N^{n}\, dS+\int_{V}\vec{f}N^{n}\, dV-\int_{V}\rho\sum_{m}\frac{\partial^{2}\overrightarrow{a^{m}}}{\partial t^{2}}N^{m}N^{n}\, dV=\vec{0}.
\end{equation}
We want to solve this equation for the unknown coefficients $\overrightarrow{a^{m}}$
subject to

\begin{gather}
\vec{u}=u^{o}\overrightarrow{}\text{ on }S_{u},\text{ and}\\
\underline{R}(\overrightarrow{u^{+}}-\overrightarrow{u^{-}})=\vec{d}\text{ on }S_{f},
\end{gather}



\section{Solution Method for Quasi-Static Problems}

For brevity we outline the solution method for quasi-static problems
using only index notation. In quasi-static problems we neglect the
inertial terms, so equation \eqref{eq:elasticity:integral:discretized}
reduces to
\begin{equation}
-\int_{V}\sigma_{ij}N_{,j}^{n}\: dV+\int_{S_{T}}T_{i}N^{n}\, dS+\int_{V}f_{i}N^{n}\, dV=\vec{0}.
\end{equation}
As a result, time-dependence only enters through the constitutive
relationships and the loading conditions. We consider the deformation
at time $t+\Delta t$,
\begin{equation}
-\int_{V}\sigma_{ij}(t+\Delta t)N_{,j}^{n}\: dV+\int_{S_{T}}T_{i}(t+\Delta t)N^{n}\, dS+\int_{V}f_{i}(t+\Delta t)N^{n}\, dV=\vec{0}.\label{eq:elasticity:integral:quasistatic}
\end{equation}
We solve this equation through formulation of a linear algebraic system
of equations ($Au=b$), involving the residual ($r=b-Au$) and Jacobian
($A$). The residual is simply
\begin{equation}
r_{i}^{n}=-\int_{V}\sigma_{ij}(t+\Delta t)N_{,j}^{n}\: dV+\int_{S_{T}}T_{i}(t+\Delta t)N^{n}\, dS+\int_{V}f_{i}(t+\Delta t)N^{n}\, dV.
\end{equation}
We employ numerical quadrature in the finite-element discretization
and replace the integrals with sums over the cells and quadrature
points,
\begin{multline}
r_{i}^{n}=-\sum_{\text{vol cells}}\sum_{\text{quad pts}}\sigma_{ij}(x_{q},t+\Delta t)N_{,j}^{n}(x_{q})\: w_{q}|J_{cell}(x_{q})|+\sum_{\text{vol cells}}\sum_{\text{quad pt}s}f_{i}(x_{q},t+\Delta t)N^{n}(x_{q})\, w_{q}|J_{cell}(x_{q})|\\
+\sum_{\text{tract cells}}\sum_{\text{quad pts}}T_{i}(x_{q},t+\Delta t)N^{n}(x_{q})\, w_{q}|J_{cell}(x_{q})|,
\end{multline}
where $r_{i}^{n}$ is an $nd$ vector ($d$ is the dimension of the
vector space) and $i$ is a vector space component, $x_{q}$ are the
coordinates of the quadrature points, $w_{q}$ are the weights of
the quadrature points, and $|J_{cell}(x_{q})|$ is the determinant
of the Jacobian matrix evaluated at the quadrature points associated
with mapping the reference cell to the actual cell. The quadrature
scheme for the integral over the tractions is one dimension lower
than the one used in integrating the terms for the volume cells.

In order to find the Jacobian of the system, we let
\begin{equation}
\sigma_{ij}(t+\Delta t)=\sigma_{ij}(t)+d\sigma_{ij}(t).
\end{equation}
Isolating the term associated with the increment in stresses yields

\begin{equation}
\int_{V}d\sigma_{ij}(t)N_{j}^{n}\ dV=-\int_{V}\sigma_{ij}(t)N_{,j}^{n}\: dV+\int_{S_{T}}T_{i}(t+\Delta t)N^{n}\, dS+\int_{V}f_{i}(t+\Delta t)N^{n}\, dV
\end{equation}
We associate the term on the left-hand-side with the action of the
system Jacobian on the increment of the displacement field. We approximate
the increment in stresses using linear elasticity and infinitesimal
strains,

\begin{gather}
d\sigma_{ij}(t)=C_{ijkl}(t)d\varepsilon_{kl}(t)\\
d\sigma_{ij}(t)=\frac{1}{2}C_{ijkl}(t)(du_{k.l}(t)+du_{l,k}(t))\\
d\sigma_{ij}(t)=\frac{1}{2}C_{ijkl}(t)(\sum_{m}da_{k,l}^{m}(t)N^{m}+\sum_{m}da_{l,k}^{m}(t)N^{m})
\end{gather}
Now, $d\sigma_{ij}\phi_{i,j}$ is a scalar, so it is symmetric,
\begin{equation}
d\sigma_{ij}\phi_{i,j}=d\sigma_{ji}\phi_{j,i},
\end{equation}
and we know that $d\sigma_{ij}$ is symmetric, so
\begin{equation}
d\sigma_{ij}\phi_{i,j}=d\sigma_{ij}\phi_{j,i},
\end{equation}
which means
\begin{equation}
\phi_{i,j}=\phi_{j,i},
\end{equation}
which we can write as
\begin{equation}
\phi_{i,j}=\frac{1}{2}(\phi_{i,j}+\phi_{j,i}).
\end{equation}
In terms of the basis functions, we have

\begin{equation}
\sum_{n}c_{i}^{n}N_{,j}^{n}=\frac{1}{2}(\sum_{n}c_{i}^{n}N_{,j}^{n}+\sum_{n}c_{j}^{n}N_{,i}^{n}).
\end{equation}
Combining these expressions for the increment in stresses and making
use of the symmetry of the weighting functions, we find the system
Jacobian is

\begin{equation}
A_{ij}^{nm}=\int_{V}\frac{1}{4}C_{ijkl}(N_{,l}^{m}+N_{,k}^{m})(N_{,j}^{n}+N_{,i}^{n})\ dV.
\end{equation}
We employ numerical quadrature in the finite-element discretization
and replace the integral with a sum over the cells and quadrature
points,
\begin{equation}
A_{ij}^{nm}=\sum_{\text{vol cells}}\sum_{\text{quad pts}}\frac{1}{4}C_{ijkl}(N_{,l}^{m}(x_{q})+N_{,k}^{m}(x_{q}))(N_{,j}^{n}(x_{q})+N_{,i}^{n}(x_{q}))w_{q}|J_{cell}(x_{q}).
\end{equation}



\section{Solution Method for Dynamic Problems}

For brevity we outline the solution method for dynamic problems using
only index notation. Time-dependence enters through the constitutive
relationships, loading conditions, and the inertial terms. We consider
the deformation at time $t$,
\begin{equation}
-\int_{V}\sigma_{ij}(t)N_{,j}^{n}\: dV+\int_{S_{T}}T_{i}(t)N^{n}\, dS+\int_{V}f_{i}(t)N^{n}\, dV-\int_{V}\rho\sum_{m}\ddot{a}_{i}^{m}(t)N^{m}N^{n}\ dV=\vec{0}.\label{eq:elasticity:integral:dynamic:t}
\end{equation}
We solve this equation through formulation of a linear algebraic system
of equations ($Au=b$), involving the residual ($r=b-Au$) and Jacobian
($A$). The residual is simply
\begin{equation}
r_{i}^{n}=-\int_{V}\sigma_{ij}(t)N_{,j}^{n}\: dV+\int_{S_{T}}T_{i}(t)N^{n}\, dS+\int_{V}f_{i}(t)N^{n}\, dV-\int_{V}\rho\sum_{m}\ddot{a}_{i}^{m}(t)N^{m}N^{n}\ dV.
\end{equation}
We employ numerical quadrature in the finite-element discretization
and replace the integrals with sums over the cells and quadrature
points,
\begin{multline}
r_{i}^{n}=-\sum_{\text{vol cells}}\sum_{\text{quad pts}}\sigma_{ij}(x_{q},t)N^{n}(x_{q})\: w_{q}|J_{cell}(x_{q})|+\sum_{\text{vol cells}}\sum_{\text{quad pt}s}f_{i}(x_{q},t)N^{n}(x_{q})\, w_{q}|J_{cell}(x_{q})|\\
+\sum_{\text{tract cells}}\sum_{\text{quad pts}}T_{i}(x_{q},t)N^{n}(x_{q})\, w_{q}|J_{cell}(x_{q})|-\sum_{\text{vol cells}}\sum_{\text{quad pts}}\rho\sum_{m}\ddot{a}_{i}^{m}(t)N^{m}N^{n}\ w_{q|J_{cell}(x_{q})},
\end{multline}
where $x_{q}$ are the coordinates of the quadrature points, $w_{q}$
are the weights of the quadrature points, and $|J_{cell}(x_{q})|$
is the determinant of the Jacobian matrix evaluated at the quadrature
points associated with mapping the reference cell to the actual cell.
The quadrature scheme for the integral over the tractions is one dimension
lower than the one used in integrating the terms for the volume cells. 

We find the system Jacobian matrix by making use of the temporal discretization
and isolating the term for the increment in the displacement field
at time $t$. Using the central difference method to approximate the
acceleration (and velocity),
\begin{gather}
\ddot{u}_{i}(t)=\frac{1}{\Delta t^{2}}\left(u_{i}(t+\Delta t)-2u_{i}(t)+u_{i}(t-\Delta t)\right)\\
\dot{u}_{i}(t)=\frac{1}{2\Delta t}\left(u_{i}(t+\Delta t)-u_{i}(t-\Delta t)\right)
\end{gather}
and writing the displacement at time $t+\Delta t$ in terms of the
displacement at $t$ (for consistency with the displacement increment
quasi-static formulation),
\begin{gather}
u_{i}(t+\Delta t)=u_{i}(t)+du_{i}(t),\\
\ddot{u}_{i}(t)=\frac{1}{\Delta t^{2}}\left(du_{i}(t)-u_{i}(t)+u_{i}(t-\Delta t)\right),\\
\dot{u}_{i}(t)=\frac{1}{2\Delta t}\left(du_{i}(t)+u_{i}(t)-u_{i}(t-\Delta t)\right).
\end{gather}
Substituting into equation \eqref{eq:elasticity:integral:dynamic:t}
yields
\begin{multline}
\frac{1}{\Delta t^{2}}\int_{V}\rho\sum_{m}da_{i}^{m}(t)N^{m}N^{n}\ dV=-\int_{V}\sigma_{ij}N_{,j}^{n}\: dV+\int_{S_{T}}T_{i}N^{n}\, dS+\int_{V}f_{i}N^{n}\, dV\\
-\frac{1}{\Delta t^{2}}\int_{V}\rho\sum_{m}(a_{i}^{m}(t)-a_{i}^{m}(t-\Delta t))N^{m}N^{n}\ dV.
\end{multline}
Thus, the Jacobian for the system is
\begin{equation}
A_{ij}^{nm}=\delta_{ij}\frac{1}{\Delta t^{2}}\int_{V}\rho N^{m}N^{n}\ dV,
\end{equation}
and using numerical quadrature in the finite-element discretization
to replace the integrals with sums over the cells and quadrature points,

\begin{equation}
A_{ij}^{nm}=\delta_{ij}\frac{1}{\Delta t^{2}}\sum_{\text{vol cells}}\sum_{\text{quad pts}}\rho(x_{q})N^{m}(x_{q})N^{n}(x_{q}),
\end{equation}
where $A_{ij}^{mn}$ is a $nd$ by $md$ matrix ($d$ is the dimension
of the vector space), $m$ and $n$ refer to the basis functions and
$i$ and $j$ are vector space components. We consider the contributions
associated with the fault in section \ref{sec:fault} and with absorbing
boundaries is section \ref{sec:absorbing:boundaries}.


\section{Small (Finite) Strain Formulation\label{sec:Small-Strain-Formulation}}

In some crustal deformation problems sufficient deformation may occur
that the assumptions associated with infinitesimal strains no longer
hold. This is often the case for problems when one wants to include
the effects of gravitational body forces on vertical deformation.
In such cases we want to account for both rigid body motion and small
strains. We use a total Lagrangian formulation (quantities are associated
with the undeformed configuration) based on the one presented by Bathe
\cite{Bathe:1995}.

Starting from the governing equation, written for the deformed configuration
(denoted by the subscript $t$), we have
\begin{equation}
\int(\nabla_{t}\cdot\underline{\sigma})\cdot\vec{\phi}\: dV_{t}+\int_{Vt}\overrightarrow{f_{t}}\cdot\overrightarrow{\phi}\, dV_{t}-\int_{Vt}\rho_{t}\frac{\partial^{2}\overrightarrow{u}}{\partial t^{2}}\cdot\overrightarrow{\phi}\, dV_{t}=0.\label{eq:governing:equation:deformed}
\end{equation}
For the total Lagrangian formulation we want to transform these integrals
over the deformed configuration to integrals over the undeformed configuration.
We require that the deformed and undeformed configurations use the
same coordinate system (origin and orientation). Conservation of mass
requires that $\rho\, dV_{t}=\rho_{0}\, dV_{0}$. We define the body
force as a force per unit volume that does not depend on the configuration,
which leads to $\vec{f}_{t}\, dV_{t}=\vec{f}_{0}\, dV_{0}$.

The Green-Lagrange strain provides a measure of the strain relative
to the original, undeformed configuration.
\begin{gather}
\underline{\epsilon}=\frac{1}{2}(\nabla u_{i,j}+(\nabla u)^{T}+u_{k,i}u_{k,j}),\text{ or}\\
\underline{\epsilon}=\underline{X}_{0}^{T}\underline{\, X}_{0}-\underline{I},\text{ where}\\
\underline{X_{0}}=\frac{\partial}{\partial x_{j}}(\vec{x}(0)+\vec{u}(t)),
\end{gather}
and $\underline{X}$ is the deformation gradient tensor. The second
Piola-Kirchhoff stress tensor, $\underline{S}$, is the work conjugate
of the Green-Lagrange strain tensor. As a result, they are related
through the elasticity constants,

\begin{equation}
\underline{S}=\underline{C\,}\underline{\varepsilon},
\end{equation}
in the same manner as the Cauchy stress is related to the infinitesimal
strain. The Cauchy stress is related to the second Piola-Kirchoff
stress through the deformation gradient tensor,
\begin{equation}
\underline{\sigma}=\frac{1}{|\underline{X}_{0}|}\underline{\, X}_{0}\,\underline{S\,}\underline{X}_{0}^{T},
\end{equation}
 where $\mathit{det}(\underline{X}_{0})=|\underline{X}_{0}|$. Additionally,
the first Piola-Kirhoff stress is define to be 
\begin{equation}
\underline{P}=\underline{S\,}\underline{X}_{0}^{T}.
\end{equation}


Applying the divergence theorem, making use of the fact that $dV_{t}=|\underline{X}_{0}|\, dV_{0}$,
and recognizing that the gradient in the deformed configuration is
related to the gradient in the undeformed configuration through the
deformation gradient tensor, we can show that
\begin{equation}
\int_{V_{t}}\nabla_{t}\cdot\underline{\sigma}\cdot\vec{\phi}\: dV_{t}=-\int_{V_{0}}\underline{P}:\nabla\overrightarrow{\phi}\, dV_{0}+\int_{S_{0}}\overrightarrow{T_{0}}\cdot\overrightarrow{\phi}\, dS_{0},
\end{equation}
where we assume the the tractions on the boundary do not depend on
the configuration. That is, the normal and share traction components
are defined in terms of the undeformed configuration. Incorporating
the other relationships between the underformed and deformed configurations
allows us to rewrite Equation \ref{eq:governing:equation:deformed}
in the undeformed configuration,
\[
-\int_{V_{0}}\underline{P}:\nabla\overrightarrow{\phi}\, dV_{0}+\int_{S_{0}}\overrightarrow{T_{0}}\cdot\overrightarrow{\phi}\, dS_{0}+\int_{V_{0}}\overrightarrow{f_{0}}\cdot\overrightarrow{\phi}\, dV_{0}-\int_{V_{0}}\rho_{0}\frac{\partial^{2}\overrightarrow{u}}{\partial t^{2}}\cdot\overrightarrow{\phi}\, dV_{0}=0.
\]



\subsection{Quasi-static Problems}

The system Jacobian for quasi-static problems includes terms associated
with elasticity. For the small strain formulation, we write the elasticity
term at time $t+\Delta t$ and consider the first terms of the Taylor
series expansion,
\begin{equation}
\int_{v}S_{ij}(t+\Delta t)\delta\varepsilon_{ij}(t+\Delta t)\: dV=\int_{V}(S_{ij}(t)\delta\varepsilon_{ij}(t)+dS_{ij}(t)\delta\varepsilon_{ij}(t)+S_{ij}(t)d\delta\varepsilon_{ij}(t))\: dV.
\end{equation}
We approximate the increment in the stress tensor using the elastic
constants,
\begin{equation}
dS_{ij}=C_{ijkl}d\varepsilon_{kl},
\end{equation}
and the increment in the ``virtual'' strain via
\begin{equation}
d\delta\varepsilon_{ij}=\frac{1}{2}(du_{k,i}\delta u_{k,j}+du_{k,j}\delta u_{k,i}).
\end{equation}
We associate the system Jacobian with the terms involving the increment
in displacements. After substituting in the expressions for the increment
in the stresses and the increment in the ``virtual'' strains, we
have
\begin{equation}
A_{ij}^{nm}=\int_{V}\frac{1}{4}C_{ijkl}(N_{,k}^{m}+(\sum_{r}a_{p}^{r}N_{,l}^{r})N_{,k}^{m})(N_{,i}^{n}+(\sum_{r}a_{p}^{r}N_{,j}^{r})N_{,i}^{n})+\frac{1}{2}S_{kl}N_{,l}^{m}N_{,l}^{n}\delta_{ij}\: dV.
\end{equation}
The small strain formulation produces additional terms associated
with the elastic constants and a new term associated with the stress
tensor.


\subsection{Dynamic Problems}

The system Jacobian matrix in dynamic problems does not include any
terms associated with elasticity, so the system Jacobian matrix in
the small strain formulation matches the one used in the infinitesimal
strain formulation.
