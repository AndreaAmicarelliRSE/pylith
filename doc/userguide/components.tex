\chapter{PyLith and Spatialdata Components}
\label{cha:components}

The name of the component is followed by the full path name and description.
The full path name is used when setting a component to a facility
in a \filename{.cfg} file or with command line arguments.


\section{Application components}
\begin{description}
\item [\object{PyLithApp}] \filename{pylith.apps.PyLithApp}\\
PyLith application.
\end{description}

\subsection{Problem Components}
\begin{description}
\item [\object{TimeDependent}] \filename{pylith.problems.TimeDependent}\\
Time-dependent problem.
\item [\object{GreensFns}] \filename{pylith.problems.GreensFns}\\
Static Green's function problem with slip impulses.
\item [\object{Implicit}] \filename{pylith.problems.Implicit}\\
Implicit time stepping for static and quasi-static simulations with
infinitesimal strains.
\item [\object{ImplicitLgDeform}] \filename{pylith.problems.ImplicitLgDeform}\\
Implicit time stepping for static and quasi-static simulations including
the effects of rigid body motion and small strains.
\item [\object{Explicit}] \filename{pylith.problems.Explicit}\\
Explicit time stepping for dynamic simulations with a lumped system
Jacobian matrix.
\item [\object{ExplicitLgDeform}] \filename{pylith.problems.ExplicitLgDeform}\\
Explicit time stepping for dynamic simulations including the effects
of rigid body motion and small strains with a lumped system Jacobian
matrix.
\item [\object{ExplicitTri3}] \filename{pylith.problems.ExplicitTri3}\\
Optimized elasticity formulation for linear triangular cells and one
quadrature point for explicit time stepping in dynamic simulations.
\item [\object{ExplicitTet4}] \filename{pylith.problems.ExplicitTet4}\\
Optimized elasticity formulation for linear tetrahedral cells and
one quadrature point for explicit time stepping in dynamic simulations.
\item [\object{SolverLinear}] \filename{pylith.problems.SolverLinear}\\
Linear PETSc solver (KSP).
\item [\object{SolverNonlinear}] \filename{pylith.problems.SolverNonlinear}\\
Nonlinear PETSc solver (SNES).
\item [\object{SolverLumped}] \filename{pylith.problems.SolverLumped}\\
Built-in simple, optimized solver for solving systems with a lumped
Jacobian.
\item [\object{TimeStepUniform}] \filename{pylith.problems.TimeStepUniform}\\
Uniform time stepping.
\item [\object{TimeStepAdapt}] \filename{pylith.problems.TimeStepAdapt}\\
Adaptive time stepping (time step selected based on estimated stable
time step).
\item [\object{TimeStepUser}] \filename{pylith.problems.TimeStepUser}\\
User defined time stepping (variable time step set by user).
\end{description}

\subsection{Utility Components}
\begin{description}
\item [\object{NullComponent}] \filename{pylith.utils.NullComponent}\\
Null component used to set a facility to an empty value.
\item [\object{EventLogger}] \filename{pylith.utils.EventLogger}\\
PETSc event logger.
\item [\object{VTKDataReader}] \filename{pylith.utils.VTKDataReader}\\
Data reader for VTK files, requires TVTK Enthought package available
from \url{https://github.com/enthought/mayavi}.
\end{description}

\subsection{Topology Components}
\begin{description}
\item [\object{Distributor}] \filename{pylith.topology.Distributor}\\
Distributor of mesh among processors in parallel simulations.
\item [\object{JacobianViewer}] \filename{pylith.topology.JacobianViewer}\\
Viewer for writing Jacobian sparse matrix to file.
\item [\object{MeshGenerator}] \filename{pylith.topology.MeshGenerator}\\
Mesh generator/importer.
\item [\object{MeshImporter}] \filename{pylith.topology.MeshImporter}\\
Mesh importer/reader.
\item [\object{MeshRefiner}] \filename{pylith.topology.MeshRefiner}\\
Default (null) mesh refinement object that does not refine the mesh.
\item [\object{RefineUniform}] \filename{pylith.topology.RefineUniform}\\
Uniform global mesh refinement.
\item [\object{ReverseCuthillMcKee}] \filename{pylith.topology.ReverseCuthillMcKee}\\
Object used to manage reordering cells and vertices using the reverse
Cuthill-McKee algorithm.
\end{description}

\subsection{Material Components}
\begin{description}
\item [\object{ElasticPlaneStrain}] \filename{pylith.materials.ElasticPlaneStrain}\\
Linearly elastic 2D bulk constitutive model with plane strain ($\epsilon_{zz}=0$).
\item [\object{ElasticPlaneStress}] \filename{pylith.materials.ElasticPlaneStress}\\
Linearly elastic 2D bulk constitutive model with plane stress ($\sigma_{zz}=0$).
\item [\object{ElasticIsotropic3D}] \filename{pylith.materials.ElasticIsotropic3D}\\
Linearly elastic 3D bulk constitutive model.
\item [\object{MaxwellIsotropic3D}] \filename{pylith.materials.MaxwellIsotropic3D}\\
Linear Maxwell viscoelastic bulk constitutive model.
\item [\object{MaxwellPlaneStrain}] \filename{pylith.materials.MaxwellPlaneStrain}\\
Linear Maxwell viscoelastic bulk constitutive model for plane strain
problems.
\item [\object{GenMaxwellIsotropic3D}] \filename{pylith.materials.GenMaxwellIsotropic3D}\\
Generalized Maxwell viscoelastic bulk constitutive model.
\item [\object{GenMaxwellPlaneStrain}] \filename{pylith.materials.GenMaxwellPlaneStrain}\\
Generalized Maxwell viscoelastic bulk constitutive model for plane
strain problems.
\item [\object{PowerLaw3D}] \filename{pylith.materials.PowerLaw3D}\\
Power-law viscoelastic bulk constitutive model.
\item [\object{PowerLawPlaneStrain}] \filename{pylith.materials.PowerLawPlaneStrain}\\
Power-law viscoelastic bulk constitutive model for plane strain problems.
\item [\object{DruckerPrage3D}] \filename{pylith.materials.DruckerPrager3D}\\
Drucker-Prager elastoplastic bulk constitutive model.
\item [\object{DruckerPragePlaneStrain}] \filename{pylith.materials.DruckerPragerPlaneStrain}\\
Drucker-Prager elastoplastic bulk constitutive model for plane strain
problems.
\item [\object{Homogeneous}] \filename{pylith.materials.Homogeneous}\\
Container with a single bulk material.
\end{description}

\subsection{Boundary Condition Components}
\begin{description}
\item [\object{AbsorbingDampers}] \filename{pylith.bc.AbsorbingDampers}\\
Absorbing boundary condition using simple dashpots.
\item [\object{DirichletBC}] \filename{pylith.bc.DirichletBC}\\
Dirichlet (prescribed displacements) boundary condition for a set
of points.
\item [\object{DirichletBoundary}] \filename{pylith.bc.DirichletBoundary}\\
Dirichlet (prescribed displacements) boundary condition for a set
of points associated with a boundary surface.
\item [\object{Neumann}] \filename{pylith.bc.Neumann}\\
Neumann (traction) boundary conditions applied to a boundary surface.
\item [\object{PointForce}] \filename{pylith.bc.PointForce}\\
Point forces applied to a set of vertices.
\item [\object{ZeroDispDB}] \filename{pylith.bc.ZeroDispDB}\\
Specialized UniformDB with uniform zero displacements at all degrees
of freedom.
\end{description}

\subsection{Fault Components}
\begin{description}
\item [\object{FaultCohesiveKin}] \filename{pylith.faults.FaultCohesiveKin}\\
Fault surface with kinematic (prescribed) slip implemented using cohesive
elements.
\item [\object{FaultCohesiveDyn}] \filename{pylith.faults.FaultCohesiveDyn}\\
Fault surface with dynamic (friction) slip implemented using cohesive
elements.
\item [\object{FaultCohesiveImpulses}] \filename{pylith.faults.FaultCohesiveImpulses}\\
Fault surface with Green's functions slip impulses implemented using
cohesive elements.
\item [\object{EqKinSrc}] \filename{pylith.faults.EqKinSrc}\\
Kinematic (prescribed) slip earthquake rupture.
\item [\object{SingleRupture}] \filename{pylith.faults.SingleRupture}\\
Container with one kinematic earthquake rupture.
\item [\object{StepSlipFn}] \filename{pylith.faults.StepSlipFn}\\
Step function slip-time function.
\item [\object{ConstRateSlipFn}] \filename{pylith.faults.ConstRateSlipFn}\\
Constant slip rate slip-time function.
\item [\object{BruneSlipFn}] \filename{pylith.faults.BruneSlipFn}\\
Slip-time function where slip rate is equal to Brune's far-field slip
function.
\item [\object{LiuCosSlipFn}] \filename{pylith.faults.LiuCosSlipFn}\\
Slip-time function composed of three sine/cosine functions. Similar
to Brune's far-field time function but with more abrupt termination
of slip.
\item [\object{TimeHistorySlipFn}] \filename{pylith.faults.TimeHistorySlipFn}\\
Slip-time function with a user-defined slip time function.
\item [\object{TractPerturbation}] \filename{pylith.faults.TractPerturbation}\\
Prescribed traction perturbation applied to fault with constitituve
model in additional to tractions from deformation (generally used
to nucleate a rupture).
\end{description}

\subsection{Friction Components}
\begin{description}
\item [\object{StaticFriction}] \filename{pylith.friction.StaticFriction}\\
Static friction fault constitutive model.
\item [\object{SlipWeakening}] \filename{pylith.friction.SlipWeakening}\\
Linear slip-weakening friction fault constitutive model.
\item [\object{RateStateAgeing}] \filename{pylith.friction.RateStateAgeing}\\
Dieterich-Ruina rate and state friction with ageing law state variable
evolution.
\item [\object{TimeWeakening}] \filename{pylith.friction.TimeWeakening}\\
Linear time-weakening friction fault constitutive model.
\end{description}

\subsection{Discretization Components}
\begin{description}
\item [\object{FIATLagrange}] \filename{pylith.feassemble.FIATLagrange}\\
Finite-element basis functions and quadrature rules for a Lagrange
reference finite-element cell (point, line, quadrilateral, or hexahedron)
using FIAT. The basis functions are constructed from the tensor produce
of 1D Lagrange reference cells.
\item [\object{FIATSimplex}] \filename{pylith.feassemble.FIATSimplex}\\
Finite-element basis functions and quadrature rules for a simplex
finite-element cell (point, line, triangle, or tetrahedron) using
FIAT.
\end{description}

\subsection{Output Components}
\begin{description}
\item [\object{MeshIOAscii}] \filename{pylith.meshio.MeshIOAscii}\\
Reader for simple mesh ASCII files.
\item [\object{MeshIOCubit}] \filename{pylith.meshio.MeshIOCubit}\\
Reader for CUBIT Exodus files.
\item [\object{MeshIOLagrit}] \filename{pylith.meshio.MeshIOLagrit}\\
Reader for LaGriT GMV/Pset files.
\item [\object{OutputManager}] \filename{pylith.meshio.OutputManager}\\
General output manager for mesh information and data.
\item [\object{OutputSoln}] \filename{pylith.meshio.OutputSoln}\\
Output manager for solution data.
\item [\object{OutputSolnSubset}] \filename{pylith.meshio.OutputSolnSubset}\\
Output manager for solution data over a submesh.
\item [\object{OutputSolnPoints}] \filename{pylith.meshio.OutputSolnPoints}\\
Output manager for solution data at arbitrary points in the domain.
\item [\object{OutputDirichlet}] \filename{pylith.meshio.OutputDirichlet}\\
Output manager for Dirichlet boundary condition information over a
submesh.
\item [\object{OutputNeumann}] \filename{pylith.meshio.OutputNeumann}\\
Output manager for Neumann boundary condition information over a submesh.
\item [\object{OutputFaultKin}] \filename{pylith.meshio.OutputFaultKin}\\
Output manager for fault with kinematic (prescribed) earthquake ruptures.
\item [\object{OutputFaultDyn}] \filename{pylith.meshio.OutputFaultDyn}\\
Output manager for fault with dynamic (friction) earthquake ruptures.
\item [\object{OutputFaultImpulses}] \filename{pylith.meshio.OutputFaultImpulses}\\
Output manager for fault with static slip impulses.
\item [\object{OutputMatElastic}] \filename{pylith.meshio.OutputMatElastic}\\
Output manager for bulk constitutive models for elasticity.
\item [\object{SingleOutput}] \filename{pylith.meshio.SingleOutput}\\
Container with single output manger.
\item [\object{PointsList}] \filename{pylith.meshio.PointsList}\\
Manager for text file container points for \filename{OutputSolnPoints}.
\item [\object{DataWriterVTK}] \filename{pylith.meshio.DataWriterVTK}\\
Writer for output to VTK files.
\item [\object{DataWriterHDF5}] \filename{pylith.meshio.DataWriterHDF5}\\
Writer for output to HDF5 files.
\item [\object{DataWriterHDF5Ext}] \filename{pylith.meshio.DataWriterHDF5Ext}\\
Writer for output to HDF5 files with datasets written to external
raw binary files.
\item [\object{CellFilterAvg}] \filename{pylith.meshio.CellFilterAvg}\\
Filter that averages information over quadrature points of cells.
\item [\object{VertexFilterVecNorm}] \filename{pylith.meshio.VertexFilterVecNorm}\\
Filter that computes magnitude of vectors for vertex fields.
\end{description}

\section{Spatialdata Components}


\subsection{Coordinate System Components}
\begin{description}
\item [\object{CSCart}] \filename{spatialdata.geocoords.CSCart}\\
Cartesian coordinate system (1D, 2D, or 3D).
\item [\object{CSGeo}] \filename{spatialdata.geocoords.CSGeo}\\
Geographic coordinate system.
\item [\object{CSGeoProj}] \filename{spatialdata.geocoords.CSGeoProj}\\
Coordinate system associated with a geographic projection.
\item [\object{CSGeoLocalCart}] \filename{spatialdata.geocoords.CSGeoLocalCart}\\
Local, georeferenced Cartesian coordinate system.
\item [\object{Projector}] \filename{spatialdata.geocoords.Projector}\\
Geographic projection.
\item [\object{Converter}] \filename{spatialdata.geocoords.Converter}\\
Converter for transforming coordinates of points from one coordinate
system to another.
\end{description}

\subsection{Spatial database Components}
\begin{description}
\item [\object{UniformDB}] \filename{spatialdata.spatialdb.UniformDB}\\
Spatial database with uniform values.
\item [\object{SimpleDB}] \filename{spatialdata.spatialdb.SimpleDB}\\
Simple spatial database that defines fields using a point cloud. Values
are determined using a nearest neighbor search or linear interpolation
in 0D, 1D, 2D, or 3D.
\item [\object{SimpleIOAscii}] \filename{spatialdata.spatialdb.SimpleIOAscii}\\
Reader/writer for simple spatial database files.
\item [\object{SCECCVMH}] \filename{spatialdata.spatialdb.SCECCVMH}\\
Spatial database interface to the SCEC CVM-H (seismic velocity model).
\item [\object{CompositeDB}] \filename{spatialdata.spatialdb.CompositeDB}\\
Spatial database composed from multiple other spatial databases.
\item [\object{TimeHistory}] \filename{spatialdata.spatialdb.TimeHistory}\\
Time history for temporal variations of a parameter.
\item [\object{GravityField}] \filename{spatialdata.spatialdb.GravityField}\\
Spatial database providing vector for body forces associated with
gravity.
\end{description}

\subsection{Nondimensionalization components}
\begin{description}
\item [\object{Nondimensional}] \filename{spatialdata.units.Nondimensional}\\
Nondimensionalization of length, time, and pressure scales.
\item [\object{NondimensionalElasticDynamic}] \filename{spatialdata.units.NondimensionalElasticDynamic}\\
Nondimensionalization of scales for dynamic problems.
\item [\object{NondimensionalElasticQuasistatic}] \filename{spatialdata.units.NondimensionalElasticQuasistatic}\\
Nondimensionalization of scales for quasi-static problems.
\end{description}

