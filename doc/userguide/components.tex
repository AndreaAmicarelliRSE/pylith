
\chapter{\label{cha:components}PyLith and Spatialdata Components}

The name of the component is followed by the full path name and description.
The full path name is used when setting a component to a facility
in a \texttt{.cfg} file or with command line arguments.


\section{Application components}
\begin{description}
\item [{\texttt{PyLithApp}}] \texttt{pylith.apps.PyLithApp}\\
PyLith application.
\end{description}

\subsection{Problem Components}
\begin{description}
\item [{\texttt{TimeDependent}}] \texttt{pylith.problems.TimeDependent}\\
Time-dependent problem.
\item [{\texttt{GreensFns}}] \texttt{pylith.problems.GreensFns}\\
Static Green's function problem with slip impulses.
\item [{\texttt{Implicit}}] \texttt{pylith.problems.Implicit}\\
Implicit time stepping for static and quasi-static simulations with
infinitesimal strains.
\item [{\texttt{ImplicitLgDeform}}] \texttt{pylith.problems.ImplicitLgDeform}\\
Implicit time stepping for static and quasi-static simulations including
the effects of rigid body motion and small strains.
\item [{\texttt{Explicit}}] \texttt{pylith.problems.Explicit}\\
Explicit time stepping for dynamic simulations with a lumped system
Jacobian matrix.
\item [{\texttt{ExplicitLgDeform}}] \texttt{pylith.problems.ExplicitLgDeform}\\
Explicit time stepping for dynamic simulations including the effects
of rigid body motion and small strains with a lumped system Jacobian
matrix.
\item [{\texttt{ExplicitTri3}}] \texttt{pylith.problems.ExplicitTri3}\\
Optimized elasticity formulation for linear triangular cells and one
quadrature point for explicit time stepping in dynamic simulations.
\item [{\texttt{ExplicitTet4}}] \texttt{pylith.problems.ExplicitTet4}\\
Optimized elasticity formulation for linear tetrahedral cells and
one quadrature point for explicit time stepping in dynamic simulations.
\item [{\texttt{SolverLinear}}] \texttt{pylith.problems.SolverLinear}\\
Linear PETSc solver (KSP).
\item [{\texttt{SolverNonlinear}}] \texttt{pylith.problems.SolverNonlinear}\\
Nonlinear PETSc solver (SNES).
\item [{\texttt{SolverLumped}}] \texttt{pylith.problems.SolverLumped}\\
Built-in simple, optimized solver for solving systems with a lumped
Jacobian.
\item [{\texttt{TimeStepUniform}}] \texttt{pylith.problems.TimeStepUniform}\\
Uniform time stepping.
\item [{\texttt{TimeStepAdapt}}] \texttt{pylith.problems.TimeStepAdapt}\\
Adaptive time stepping (time step selected based on estimated stable
time step).
\item [{\texttt{TimeStepUser}}] \texttt{pylith.problems.TimeStepUser}\\
User defined time stepping (variable time step set by user).
\end{description}

\subsection{Utility Components}
\begin{description}
\item [{\texttt{NullComponent}}] \texttt{pylith.utils.NullComponent}\\
Null component used to set a facility to an empty value.
\item [{\texttt{EventLogger}}] \texttt{pylith.utils.EventLogger}\\
PETSc event logger.
\item [{\texttt{VTKDataReader}}] \texttt{pylith.utils.VTKDataReader}\\
Data reader for VTK files, requires TVTK Enthought package available
from \texttt{}~\\
\texttt{https://github.com/enthought/mayavi.}
\end{description}

\subsection{Topology Components}
\begin{description}
\item [{\texttt{Distributor}}] \texttt{pylith.topology.Distributor}\\
Distributor of mesh among processors in parallel simulations.
\item [{\texttt{JacobianViewer}}] \texttt{pylith.topology.JacobianViewer}\\
Viewer for writing Jacobian sparse matrix to file.
\item [{\texttt{MeshGenerator}}] \texttt{pylith.topology.MeshGenerator}\\
Mesh generator/importer.
\item [{\texttt{MeshImporter}}] \texttt{pylith.topology.MeshImporter}\\
Mesh importer/reader.
\item [{\texttt{MeshRefiner}}] \texttt{pylith.topology.MeshRefiner}\\
Default (null) mesh vrefinement object that does not vrefine the mesh.
\item [{\texttt{RefineUniform}}] \texttt{pylith.topology.RefineUniform}\\
Uniform global mesh vrefinement.
\item [{\texttt{ReverseCuthillMcKee}}] \texttt{pylith.topology.ReverseCuthillMcKee}\\
Object used to manage reordering cells and vertices using the reverse
Cuthill-McKee algorithm.
\end{description}

\subsection{Material Components}
\begin{description}
\item [{\texttt{ElasticStrain1D}}] \texttt{pylith.materials.ElasticStrain1D}\\
Linearly elastic 1D bulk constitutive model with 1D strain ($\epsilon_{yy}=\epsilon_{zz}=0$).
\item [{\texttt{ElasticStress1D}}] \texttt{pylith.materials.ElasticStress1D}\\
Linearly elastic 1D bulk constitutive model with 1D stress ($\sigma_{yy}=\sigma_{zz}=0$).
\item [{\texttt{ElasticPlaneStrain}}] \texttt{pylith.materials.ElasticPlaneStrain}\\
Linearly elastic 2D bulk constitutive model with plane strain ($\epsilon_{zz}=0$).
\item [{\texttt{ElasticPlaneStress}}] \texttt{pylith.materials.ElasticPlaneStress}\\
Linearly elastic 2D bulk constitutive model with plane stress ($\sigma_{zz}=0$).
\item [{\texttt{ElasticIsotropic3D}}] \texttt{pylith.materials.ElasticIsotropic3D}\\
Linearly elastic 3D bulk constitutive model.
\item [{\texttt{MaxwellIsotropic3D}}] \texttt{pylith.materials.MaxwellIsotropic3D}\\
Linear Maxwell viscoelastic bulk constitutive model.
\item [{\texttt{MaxwellPlaneStrain}}] \texttt{pylith.materials.MaxwellPlaneStrain}\\
Linear Maxwell viscoelastic bulk constitutive model for plane strain
problems.
\item [{\texttt{GenMaxwellIsotropic3D}}] \texttt{pylith.materials.GenMaxwellIsotropic3D}\\
Generalized Maxwell viscoelastic bulk constitutive model.
\item [{\texttt{GenMaxwellPlaneStrain}}] \texttt{pylith.materials.GenMaxwellPlaneStrain}\\
Generalized Maxwell viscoelastic bulk constitutive model for plane
strain problems.
\item [{\texttt{PowerLaw3D}}] \texttt{pylith.materials.PowerLaw3D}\\
Power-law viscoelastic bulk constitutive model.
\item [{\texttt{PowerLawPlaneStrain}}] \texttt{pylith.materials.PowerLawPlaneStrain}\\
Power-law viscoelastic bulk constitutive model for plane strain problems.
\item [{\texttt{DruckerPrage3D}}] \texttt{pylith.materials.DruckerPrager3D}\\
Drucker-Prager elastoplastic bulk constitutive model.
\item [{\texttt{DruckerPragePlaneStrain}}] \texttt{pylith.materials.DruckerPragerPlaneStrain}\\
Drucker-Prager elastoplastic bulk constitutive model for plane strain
problems.
\item [{\texttt{Homogeneous}}] \texttt{pylith.materials.Homogeneous}\\
Container with a single bulk material.
\end{description}

\subsection{Boundary Condition Components}
\begin{description}
\item [{\texttt{AbsorbingDampers}}] \texttt{pylith.bc.AbsorbingDampers}\\
Absorbing boundary condition using simple dashpots.
\item [{\texttt{DirichletBC}}] \texttt{pylith.bc.DirichletBC}\\
Dirichlet (prescribed displacements) boundary condition for a set
of points.
\item [{\texttt{DirichletBoundary}}] \texttt{pylith.bc.DirichletBoundary}\\
Dirichlet (prescribed displacements) boundary condition for a set
of points associated with a boundary surface.
\item [{\texttt{Neumann}}] \texttt{pylith.bc.Neumann}\\
Neumann (traction) boundary conditions applied to a boundary surface.
\item [{\texttt{PointForce}}] \texttt{pylith.bc.PointForce}\\
Point forces applied to a set of vertices.
\item [{\texttt{ZeroDispDB}}] \texttt{pylith.bc.ZeroDispDB}\\
Specialized UniformDB with uniform zero displacements at all degrees
of freedom.
\end{description}

\subsection{Fault Components}
\begin{description}
\item [{\texttt{FaultCohesiveKin}}] \texttt{pylith.faults.FaultCohesiveKin}\\
Fault surface with kinematic (prescribed) slip implemented using cohesive
elements.
\item [{\texttt{FaultCohesiveDyn}}] \texttt{pylith.faults.FaultCohesiveDyn}\\
Fault surface with dynamic (friction) slip implemented using cohesive
elements.
\item [{\texttt{FaultCohesiveImpulses}}] \texttt{pylith.faults.FaultCohesiveImpulses}\\
Fault surface with Green's functions slip impulses implemented using
cohesive elements.
\item [{\texttt{EqKinSrc}}] \texttt{pylith.faults.EqKinSrc}\\
Kinematic (prescribed) slip earthquake rupture.
\item [{\texttt{SingleRupture}}] \texttt{pylith.faults.SingleRupture}\\
Container with one kinematic earthquake rupture.
\item [{\texttt{StepSlipFn}}] \texttt{pylith.faults.StepSlipFn}\\
Step function slip-time function.
\item [{\texttt{ConstRateSlipFn}}] \texttt{pylith.faults.ConstRateSlipFn}\\
Constant slip rate slip-time function.
\item [{\texttt{BruneSlipFn}}] \texttt{pylith.faults.BruneSlipFn}\\
Slip-time function where slip rate is equal to Brune's far-field slip
function.
\item [{\texttt{LiuCosSlipFn}}] \texttt{pylith.faults.LiuCosSlipFn}\\
Slip-time function composed of three sine/cosine functions. Similar
to Brune's far-field time function but with more abrupt termination
of slip.
\item [{\texttt{TimeHistorySlipFn}}] \texttt{pylith.faults.TimeHistorySlipFn}\\
Slip-time function with a user-defined slip time function.
\item [{\texttt{TractPerturbation}}] \texttt{pylith.faults.TractPerturbation}\\
Prescribed traction perturbation applied to fault with constitituve
model in additional to tractions from deformation (generally used
to nucleate a rupture).
\end{description}

\subsection{Friction Components}
\begin{description}
\item [{\texttt{StaticFriction}}] \texttt{pylith.friction.StaticFriction}\\
Static friction fault constitutive model.
\item [{\texttt{SlipWeakening}}] \texttt{pylith.friction.SlipWeakening}\\
Linear slip-weakening friction fault constitutive model.
\item [{\texttt{RateStateAgeing}}] \texttt{pylith.friction.RateStateAgeing}\\
Dieterich-Ruina rate and state friction with ageing law state variable
evolution.
\item [{\texttt{TimeWeakening}}] \texttt{pylith.friction.TimeWeakening}\\
Linear time-weakening friction fault constitutive model.
\end{description}

\subsection{Discretization Components}
\begin{description}
\item [{\texttt{FIATLagrange}}] \texttt{pylith.feassemble.FIATLagrange}\\
Finite-element basis functions and quadrature rules for a Lagrange
vreference finite-element cell (point, line, quadrilateral, or hexahedron)
using FIAT. The basis functions are constructed from the tensor produce
of 1D Lagrange vreference cells.
\item [{\texttt{FIATSimplex}}] \texttt{pylith.feassemble.FIATSimplex}\\
Finite-element basis functions and quadrature rules for a simplex
finite-element cell (point, line, triangle, or tetrahedron) using
FIAT.
\end{description}

\subsection{Output Components}
\begin{description}
\item [{\texttt{MeshIOAscii}}] \texttt{pylith.meshio.MeshIOAscii}\\
Reader for simple mesh ASCII files.
\item [{\texttt{MeshIOCubit}}] \texttt{pylith.meshio.MeshIOCubit}\\
Reader for CUBIT Exodus files.
\item [{\texttt{MeshIOLagrit}}] \texttt{pylith.meshio.MeshIOLagrit}\\
Reader for LaGriT GMV/Pset files.
\item [{\texttt{OutputManager}}] \texttt{pylith.meshio.OutputManager}\\
General output manager for mesh information and data.
\item [{\texttt{OutputSoln}}] \texttt{pylith.meshio.OutputSoln}\\
Output manager for solution data.
\item [{\texttt{OutputSolnSubset}}] \texttt{pylith.meshio.OutputSolnSubset}\\
Output manager for solution data over a submesh.
\item [{\texttt{OutputSolnPoints}}] \texttt{pylith.meshio.OutputSolnPoints}\\
Output manager for solution data at arbitrary points in the domain.
\item [{\texttt{OutputDirichlet}}] \texttt{pylith.meshio.OutputDirichlet}\\
Output manager for Dirichlet boundary condition information over a
submesh.
\item [{\texttt{OutputNeumann}}] \texttt{pylith.meshio.OutputNeumann}\\
Output manager for Neumann boundary condition information over a submesh.
\item [{\texttt{OutputFaultKin}}] \texttt{pylith.meshio.OutputFaultKin}\\
Output manager for fault with kinematic (prescribed) earthquake ruptures.
\item [{\texttt{OutputFaultDyn}}] \texttt{pylith.meshio.OutputFaultDyn}\\
Output manager for fault with dynamic (friction) earthquake ruptures.
\item [{\texttt{OutputFaultImpulses}}] \texttt{pylith.meshio.OutputFaultImpulses}\\
Output manager for fault with static slip impulses.
\item [{\texttt{OutputMatElastic}}] \texttt{pylith.meshio.OutputMatElastic}\\
Output manager for bulk constitutive models for elasticity.
\item [{\texttt{SingleOutput}}] \texttt{pylith.meshio.SingleOutput}\\
Container with single output manger.
\item [{\texttt{PointsList}}] \texttt{pylith.meshio.PointsList}\\
Manager for text file container points for \texttt{OutputSolnPoints}.
\item [{\texttt{DataWriterVTK}}] \texttt{pylith.meshio.DataWriterVTK}\\
Writer for output to VTK files.
\item [{\texttt{DataWriterHDF5}}] \texttt{pylith.meshio.DataWriterHDF5}\\
Writer for output to HDF5 files.
\item [{\texttt{DataWriterHDF5Ext}}] \texttt{pylith.meshio.DataWriterHDF5Ext}\\
Writer for output to HDF5 files with datasets written to external
raw binary files.
\item [{\texttt{CellFilterAvg}}] \texttt{pylith.meshio.CellFilterAvg}\\
Filter that averages information over quadrature points of cells.
\item [{\texttt{VertexFilterVecNorm}}] \texttt{pylith.meshio.VertexFilterVecNorm}\\
Filter that computes magnitude of vectors for vertex fields.
\end{description}

\section{Spatialdata Components}


\subsection{Coordinate System Components}
\begin{description}
\item [{\texttt{CSCart}}] \texttt{spatialdata.geocoords.CSCart}\\
Cartesian coordinate system (0D, 1D, 2D, or 3D).
\item [{\texttt{CSGeo}}] \texttt{spatialdata.geocoords.CSGeo}\\
Geographic coordinate system.
\item [{\texttt{CSGeoProj}}] \texttt{spatialdata.geocoords.CSGeoProj}\\
Coordinate system associated with a geographic projection.
\item [{\texttt{CSGeoLocalCart}}] \texttt{spatialdata.geocoords.CSGeoLocalCart}\\
Local, geovreferenced Cartesian coordinate system.
\item [{\texttt{Projector}}] \texttt{spatialdata.geocoords.Projector}\\
Geographic projection.
\item [{\texttt{Converter}}] \texttt{spatialdata.geocoords.Converter}\\
Converter for transforming coordinates of points from one coordinate
system to another.
\end{description}

\subsection{Spatial database Components}
\begin{description}
\item [{\texttt{UniformDB}}] \texttt{spatialdata.spatialdb.UniformDB}\\
Spatial database with uniform values.
\item [{\texttt{SimpleDB}}] \texttt{spatialdata.spatialdb.SimpleDB}\\
Simple spatial database that defines fields using a point cloud. Values
are determined using a nearest neighbor search or linear interpolation
in 0D, 1D, 2D, or 3D.
\item [{\texttt{SimpleIOAscii}}] \texttt{spatialdata.spatialdb.SimpleIOAscii}\\
Reader/writer for simple spatial database files.
\item [{\texttt{SCECCVMH}}] \texttt{spatialdata.spatialdb.SCECCVMH}\\
Spatial database interface to the SCEC CVM-H (seismic velocity model).
\item [{\texttt{CompositeDB}}] \texttt{spatialdata.spatialdb.CompositeDB}\\
Spatial database composed from multiple other spatial databases.
\item [{\texttt{TimeHistory}}] \texttt{spatialdata.spatialdb.TimeHistory}\\
Time history for temporal variations of a parameter.
\item [{\texttt{GravityField}}] \texttt{spatialdata.spatialdb.GravityField}\\
Spatial database providing vector for body forces associated with
gravity.
\end{description}

\subsection{Nondimensionalization components}
\begin{description}
\item [{\texttt{Nondimensional}}] \texttt{spatialdata.units.Nondimensional}\\
Nondimensionalization of length, time, and pressure scales.
\item [{\texttt{NondimensionalElasticDynamic}}] \texttt{spatialdata.units.NondimensionalElasticDynamic}\\
Nondimensionalization of scales for dynamic problems.
\item [{\texttt{NondimensionalElasticQuasistatic}}] \texttt{spatialdata.units.NondimensionalElasticQuasistatic}\\
Nondimensionalization of scales for quasi-static problems.\end{description}

