\chapter{Material Models}

\section{Effective Stress Function Formulation for a Maxwell Linear
Viscoelastic Material}

\subsection{Determination of stresses}

The element stresses are
\begin{equation}
  \text{figs/ml-eq1.eps}
\end{equation}
where ?? is the total strain and $\text{ml-inlineeq1}$ is the initial
stress. In terms of the deviatoric stress,
\begin{equation}
  \text{figs/ml-eq2.eps}
\end{equation}
where
\begin{equation}
  \text{figs/ml-eq3.eps}
\end{equation}
and the mean stress and strain are given by
\begin{equation}
  \text{figs/ml-eq4.eps}
\end{equation}
Equation~(2) [REPLACE WITH REF] may also be written as
\begin{equation}
  \text{figs/ml-eq4.eps}
\end{equation}
where
\begin{equation}
  \text{figs/ml-eq6.eps}
\end{equation}
The creep strain increment is approximated using
\begin{equation}
  \text{figs/ml-eq7.eps}
\end{equation}
where, using the $\alpha$-method of time integration,
\begin{equation}
  \text{figs/ml-eq8.eps}
\end{equation}
and
\begin{equation}
  \text{figs/ml-eq9.eps}
\end{equation}
where
\begin{equation}
  \text{figs/ml-eq10.eps}
\end{equation}
and
\begin{equation}
  \text{figs/ml-eq11.eps}
\end{equation}
For a linear Maxwell viscoelastic material
\begin{equation}
  \text{figs/ml-eq12.eps}
\end{equation}
Therefore,
\begin{equation}
  \text{figs/ml-eq13.eps}
\end{equation}
Subsituting (8), (12), and (13) into (5) [REPLACE WITH REFS], we obtain
\begin{equation}
  \text{figs/ml-eq14.eps}
\end{equation}
Solving for  $\text{figs/ml-inlineeq2}$,
\begin{equation}
  \text{figs/ml-eq15.eps}
\end{equation}
In this case it is possible to solve directly for the deviatoric
stresses, and the effective stress function approach is not needed. To
obtain the total stress, we simply use
\begin{equation}
  \text{figs/ml-eq16.eps}
\end{equation}

\subsection{Tangent stress-strain relation}

It is now necessary to provide a relationship for the viscoelastic
tangent material matrix. If we use vectors composed of the stresses
and tensor strains, this relationship is
\begin{equation}
  \text{figs/ml-eq17.eps}
\end{equation}
In terms of the vectors, we have
\begin{equation}
  \text{figs/ml-eq18.eps}
\end{equation}
Therefore,
\begin{equation}
  \text{figs/ml-eq19.eps}
\end{equation}
Using the chain rule,
\begin{equation}
  \text{figs/ml-eq20.eps}
\end{equation}
From (6) [REPLACE WITH REF], we obtain
\begin{equation}
  \text{figs/ml-eq21.eps}
\end{equation}
and from (3) [REPLACE WITH REF]
\begin{equation}
  \text{figs/ml-eq22.eps}
\end{equation}
Finally, from (15) [REPLACE WITH REF], we have
\begin{equation}
  \text{figs/ml-eq23.eps}
\end{equation}
From (19) [REPLACE WITH REF], the final material matrix relating
stress and tensor strain is
\begin{equation}
  \text{figs/ml-eq24.eps}
\end{equation}
Note that the coefficient of the second matrix approaches $E/3(1+\nu)$
as $\eta$ goes to infinity. Since finite element computations
typically use engineering strain measures, the matrix that is actually
used is
\begin{equation}
  \text{figs/ml-eq25.eps}
\end{equation}
To check the results we make sure that the regular elastic
constitutive matrix is obtained for selected terms in the case where
$\eta$ goes to infinity.
\begin{equation}
  \text{figs/ml-eq26.eps}
\end{equation}
This is consistent with the regular elasticity matrix, and equation
(25) [REPLACE WITH REF] should thus be used when forming the stiffness
matrix.
