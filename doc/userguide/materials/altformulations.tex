
\chapter{\label{cha:Alternative-Formulations}Alternative Material Model Formulations}


\section{\label{sec:ViscoelasticFormulations}Viscoelastic Formulations}

The viscoelastic formulations presently used in PyLith are described
in Section \ref{sec:Viscoelastic-Materials}. In some cases there
are alternative formulations that may be used in future versions of
PyLith, and those are described here.


\subsection{\label{sub:Effective-Stress-Formulation-Maxwell}Effective Stress
Formulation for a Linear Maxwell Viscoelastic Material}

An alternative technique for solving the equations for a Maxwell viscoelastic
material is based on the effective stress formulation described in
Section \ref{sub:Effective-Stress-Formulations-Viscoelastic}. A linear
Maxwell viscoelastic material may be characterized by the same elastic
parameters as an isotropic elastic material ($E$ and $\nu$), as
well as the viscosity, $\eta$. The creep strain increment is
\begin{gather}
\underline{\Delta e}^{C}=\frac{\Delta t\phantom{}^{\tau}\underline{S}}{2\eta}\,\,.\label{eq:D1}
\end{gather}
Therefore,
\begin{gather}
\Delta\overline{e}^{C}=\frac{\Delta t\sqrt{^{\tau}J_{2}^{\prime}}}{\sqrt{3\eta}}=\frac{\Delta t\phantom{}^{\tau}\overline{\sigma}}{3\eta}\,,\,\mathrm{and}\,^{\tau}\gamma=\frac{1}{2\eta}\,\,.\label{eq:D2}
\end{gather}
Substituting Equations \ref{eq:46}, \ref{eq:D1}, and \ref{eq:D2}
into \ref{eq:43}, we obtain
\begin{gather}
^{t+\Delta t}\underline{S}=\frac{1}{a_{E}}\left\{ ^{t+\Delta t}\underline{e}^{\prime}-\frac{\Delta t}{2\eta}\left[(1-\alpha)^{t}\underline{S}+\alpha\phantom{}^{t+\Delta t}\underline{S}\right]\right\} +\underline{S}^{I}\,\,.\label{eq:D3}
\end{gather}
Solving for $^{t+\Delta t}\underline{S}$,
\begin{gather}
^{t+\Delta t}\underline{S}=\frac{1}{a_{E}+\frac{\alpha\Delta t}{2\eta}}\left[^{t+\Delta t}\underline{e}^{\prime}-\frac{\Delta t}{2\eta}(1-\alpha)^{t}\underline{S}+\frac{1+\mathrm{\nu}}{E}\underline{S}^{I}\right]\,\,.\label{eq:D4}
\end{gather}
In this case it is possible to solve directly for the deviatoric stresses,
and the effective stress function approach is not needed. To obtain
the total stress, we simply use
\begin{gather}
^{t+\Delta t}\sigma_{ij}=\phantom{}^{t+\Delta t}S_{ij}+\frac{\mathit{1}}{a_{m}}\left(\,^{t+\Delta t}\theta-\theta^{I}\right)\delta_{ij}+P^{I}\delta_{ij}\,\,.\label{eq:D5}
\end{gather}


To compute the viscoelastic tangent material matrix relating stress
and strain, we need to compute the first term in Equation \ref{eq:58}.
From Equation \ref{eq:D4}, we have
\begin{gather}
\frac{\partial\phantom{}^{t+\Delta t}S_{i}}{\partial\phantom{}^{t+\Delta t}e_{k}^{\prime}}=\frac{\delta_{ik}}{a_{E}+\frac{\alpha\Delta t}{2\eta}}\,\,.\label{eq:D12}
\end{gather}
Using this, along with Equations \ref{eq:58}, \ref{eq:59}, and \ref{eq:60},
the final material matrix relating stress and tensor strain is
\begin{gather}
C_{ij}^{VE}=\frac{1}{3a_{m}}\left[\begin{array}{cccccc}
1 & 1 & 1 & 0 & 0 & 0\\
1 & 1 & 1 & 0 & 0 & 0\\
1 & 1 & 1 & 0 & 0 & 0\\
0 & 0 & 0 & 0 & 0 & 0\\
0 & 0 & 0 & 0 & 0 & 0\\
0 & 0 & 0 & 0 & 0 & 0
\end{array}\right]+\frac{1}{3\left(a_{E}+\frac{\alpha\Delta t}{2\eta}\right)}\left[\begin{array}{cccccc}
2 & -1 & -1 & 0 & 0 & 0\\
-1 & 2 & -1 & 0 & 0 & 0\\
-1 & -1 & 2 & 0 & 0 & 0\\
0 & 0 & 0 & 3 & 0 & 0\\
0 & 0 & 0 & 0 & 3 & 0\\
0 & 0 & 0 & 0 & 0 & 3
\end{array}\right]\,.\label{eq:D13}
\end{gather}
Note that the coefficient of the second matrix approaches $E/3(1+\nu)=1/3a_{E}$
as $\eta$ goes to infinity. To check the results we make sure that
the regular elastic constitutive matrix is obtained for selected terms
in the case where $\eta$ goes to infinity.
\begin{gather}
C_{11}^{E}=\frac{E(1-\nu)}{(1+\nu)(1-2\nu)}\,\,\nonumber \\
C_{12}^{E}=\frac{E\nu}{(1+\nu)(1-2\nu)}\,.\label{eq:D14}\\
C_{44}^{E}=\frac{E}{1+\nu}\,\,\nonumber 
\end{gather}
This is consistent with the regular elasticity matrix, and Equation
\ref{eq:D13} should thus be used when forming the stiffness matrix.
We do not presently use this formulation, but it may be included in
future versions.
