\chapter{Glossary}
\label{cha:glossary}

\section{Pyre}
\begin{description}
\item [\facility{component}] Basic building block of a Pyre
  application. A component may be built-up from smaller building
  blocks, where simple data types are called properties and data
  structures and objects are called facilities.  In general a
  component is a specific implementation of the functionality of a
  facility. For example, SimpleDB is a specific implementation of the
  spatial database facility. A component is generally composed of a
  Python object and a C++ object, although either one may be missing.
  We nearly always use the naming convention such that for an object
  called Foo the Python object is in file Foo.py, the C++ class
  definition is in Foo.hh, inline C++ functions are in foo.icc, the
  C++ class implementation is in Foo.cc, and the SWIG interface file
  that glues the C++ and Python code together is in Foo.i.
\item [\facility{facility}] Complex data type (object or data
  structure) building block of a component. See component.
\item [\property{property}] Simple data type (string, integer, real
  number, or boolean value) parameter for a component.
\end{description}

\section{\object{DMPlex}}

The plex construction is a representation of the topology of the
finite element mesh based upon a covering relation. For example,
segments are covered by their endpoints, faces by their bounding
edges, etc. Geometry is absent from the plex, and is represented
instead by a field with the coordinates of the vertices. Meshes can
also be understood as directed acyclic graphs, where we call the
constituents points and arrows.

\begin{description}
\item [mesh] A finite element mesh, used to partition space and
  provide support for the basis functions.
\item [cell] The highest dimensional elements of a mesh, or mesh
  entities of codimension zero.
\item [vertex] The zero dimensional mesh elements.
\item [face] Mesh elements that separate cells, or mesh entities of
  codimension one.
\item [field] A parallel section which can be completed, or made
  consistent, across process boundaries. These are used to represent
  continuum fields.
\item [section] These objects associate values in vectors to points
  (vertices, edges, faces, and cells) in a mesh. The section describes
  the offset into the vector along with the number of values
  associated with each point.
\item [dimension] The topological dimension of the mesh, meaning the
  cell dimension. It can also mean the dimension of the space in which
  the mesh is embedded, but this is properly the embedding dimension.
\item [fiber\ dimension] Dimension of the space associated with the
  field. For example, the scalar field has a fiber dimension of 1 and
  a vector field has a fiber displacement equal to the dimension of
  the mesh.
\item [cohesive\ cell] A zero volume cell inserted between any two
  cells which shared a fault face. They are prisms with a fault face
  as the base.
\item [cone] The set of entities which cover any entity in a mesh. For
  example, the cone of a triangle is its three edges.
\item [support] The set of mesh entities which are covered by any
  entity in a mesh. For example, the support of a triangle is the two
  tetrahedra it separates.
\end{description}

