\section{Post-Processing Utilities}

The PyLith distribution includes a few post-processing utilities.
These are Python scripts that are installed into the same bin directory
as the \filename{pylith} executable.


\subsection{\filename{pylith\_eqinfo}}

This utility computes the moment magnitude, seismic moment, seismic
potency, and average slip at user-specified time snapshots from PyLith
fault HDF5 output. The utility works with output from simulations
with either prescribed slip and/or spontaneous rupture. Currently,
we compute the shear modulus from a user-specified spatial database
at the centroid of the fault cells. In the future we plan to account
for lateral variations in shear modulus across the fault when calculating
the seismic moment. The Python script is a Pyre application, so its
parameters can be specified using \filename{cfg} and command line arguments
just like PyLith. The Pyre properties and facilities include:
\begin{inventory}
\propertyitem{output\_filename}{Filename for output of slip information.}
\propertyitem{faults}{Array of fault names.}
\propertyitem{filename\_pattern}{Filename pattern in C/Python format for creating
filename for each fault. Default is \filename{output/fault\_\%s.h5}.}
\propertyitem{snapshots}{Array of timestamps for slip snapshosts ([-1] means
use last time step in file, which is the default).}
\propertyitem{snapshot\_units}{Units for timestamps in array of snapshots.}
\facilityitem{db\_properties}{Spatial database for elastic properties.}
\facilityitem{coordsys}{Coordinate system associated with mesh in simulation.}
\end{inventory}

\subsection{\filename{pylith\_genxdmf}}
\label{sec:pylith:genxdmf}

This utility generates Xdmf files from HDF5 files that conform to the
layout used by PyLith. It is a simple Python script with a single
command line argument with the file pattern of HDF5 files for which
Xdmf files should be generated. Typically, it is used to regenerate
Xdmf files that get corrupted or lost due to renaming and moving. It
is also useful in updating Xdmf files when users add fields to HDF5
files during post-processing.
\begin{shell}
$ pylith_genxdmf --files=FILE_OR_FILE_PATTERN
\end{shell}
The default value for \filename{FILE\_OR\_FILE\_PATTERN} is \filename{*.h5}.

\warning{If the HDF5 files contain external datasets, then this
  utility should be run from the same relative path to the HDF5 files
  as when they were created. For example, if a PyLith simulation was
  run from directory \filename{work} and HDF5 files were generated in
  \filename{output/work}, then the utility should be run from the
  directory \filename{work}. Furthermore, a visualization tool, such
  as ParaView, should also be started from the working directory
  \filename{work}.}

% End of file
