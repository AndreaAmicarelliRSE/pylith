\section{Databases for Boundaries, Interfaces, and Material Properties}
\label{sec:spatial:databases}

Once the problem has been defined with PyLith parameters, and the
mesh information has been provided, the final step is to specify the
boundary conditions and material properties to be used. The mesh information
provides labels defining sets of vertices to which boundary conditions
or fault conditions will be applied, as well as cell labels that will
be used to define the material type of each cell. For boundary conditions,
the \filename{.cfg} file is used to associate boundary condition types
and spatial databases with each vertex group (see Chapter \vref{cha:boundary:interface:conditions}).
For materials, the \filename{.cfg} file is used to associate material
types and spatial databases with cells identified by the material
identifier (see Figure \vref{fig:material:models}).

The spatial databases define how the boundary conditions or material
property values vary spatially, and they can be arbitrarily complex.
The simplest example for a material database would be a mesh where
all the cells of a given type have uniform properties (``point''
or 0D variation). A slightly more complex case would be a mesh where
the cells of a given type have properties that vary linearly along
a given direction (``line'' or 1D variation). In more complex models,
the material properties might have different values at each point
in the mesh (``volume'' or 3D variation). This might be the case,
for example, if the material properties are provided by a database
of seismic velocities and densities. For boundary conditions the simplest
case would be where all vertices in a given group have the same boundary
condition parameters (``point'' or 0D variation). A more complex
case might specify a variation in the conditions on a given surface
(``area'' or 2D variation). This sort of condition might be used,
for example, to specify the variation of slip on a fault plane. The
examples discussed in Chapter \vref{cha:examples} also contain more
information regarding the specification and use of the spatial database
files.


\subsection{\object{SimpleDB} Spatial Database}

In most cases the default type of spatial database for faults, boundary
conditions, and materials is \object{SimpleDB}. Spatial database files
provide specification of a field over some set of points. There is
no topology associated with the points. Although multiple values can
be specified at each point with more than one value included in a
search query, the interpolation of each value will be done independently.
Time dependent variations of a field are not supported in these files.
Spatial database files can specify spatial variations over zero, one,
two, and three dimensions. Zero dimensional variations correspond
to uniform values. One-dimensional spatial variations correspond to
piecewise linear variations, which need not coincide with coordinate
axes. Likewise, two-dimensional spatial variations correspond to variations
on a planar surface (which need not coincide with the coordinate axes)
and three-dimensional spatial variations correspond to variations
over a volume. In one, two, or three dimensions, queries can use a
``nearest value'' search or linear interpolation.

The spatial database files need not provide the data using the same
coordinate system as the mesh coordinate system, provided the two
coordinate systems are compatible. Examples of compatible coordinate
systems include geographic coordinates (longitude/latitude/elevation),
and projected coordinates (e.g., coordinates in a transverse Mercator
projection). Spatial database queries use the Proj.4 Cartographic
Projections library \url{proj.maptools.org} to convert between coordinate
systems, so a large number of geographic projections are available
with support for converting between NAD27 and WGS84 horizontal datums
as well as several other frequently used datums. Because the interpolation
is done in the coordinate system of the spatial database, geographic
coordinates should only be used for very simple datasets, or undesirable
results will occur. This is especially true when the spatial database
coordinate system combines latitude, longitude, and elevation in meters
(longitude and latitude in degrees are often much smaller than elevations
in meters leading to distorted ``distance'' between locations and
interpolation).

\object{SimpleDB} uses a simple ASCII file to specify the variation of values (e.g., displacement field, slip field, physical properties) in space.  The file format is described in Section \vref{sec:format:SimpleIOAscii}.  The examples in Chapter \vref{cha:examples} use \object{SimpleDB} files to specify the values for the boundary conditions, physical properties, and fault slip.

As in the other Pyre objects, spatial database objects contain parameters
that can be set from the command line or using \filename{.cfg}
files. The properties and facilities for a spatial database are:
\begin{inventory}
\propertyitem{label}{Label for the database, which is used in diagnostic messages.}
\propertyitem{query\_type}{Type of search query to perform. Values for this
parameter are ``linear'' and ``nearest'' (default).}
\facilityitem{iohandler}{Database importer. Only one importer is implemented,
so you do not need to change this setting.}
\propertyitem{iohandler.filename}{Filename for the spatial database.}
\end{inventory}
An example of setting these in a \filename{.cfg} file is:
\begin{cfg}
<p>label</p> = Material properties
<p>query_type</p> = linear
<p>iohandler.filename</p> = mydb.spatialdb
\end{cfg}

\subsection{\object{UniformDB} Spatial Database}

The \object{SimpleDB} spatial database is quite general, but when the values
are uniform, it is often easier to use the \object{UniformDB} spatial database
instead. With the \object{UniformDB}, you specify the values directly either
on the command line or in a parameter-setting (\filename{.cfg}) file.
On the other hand, if the values are used in more than one place,
it is easier to place the values in a \object{SimpleDB} file, because they
can then be referred to using the filename of the spatial database
rather than having to repeatedly list all of the values on the command
line or in a parameter-setting (\filename{.cfg}) file. The properties
for a \object{UniformDB} are:
\begin{inventory}
\propertyitem{values}{Array of names of values in spatial database.}
\propertyitem{data}{Array of values in spatial database.}
\end{inventory}

Specify the physical properties of a linearly elastic, isotropic material
in a \filename{.cfg} file. The data values are dimensioned
with the appropriate units using Python syntax.
\begin{cfg}
<h>[pylithapp.timedependent.materials.material]</h>
<p>db_properties</p> = spatialdata.spatialdb.UniformDB ; Set the db to a UniformDB
<p>db_properties.values</p> = [vp, vs, density] ; Set the names of the values in the database
<p>db_properties.data</p> = [5773.5*m/s, 3333.3*m/s, 2700.0*kg/m**3] ; Set the values in the database}
\end{cfg}

\subsubsection{\object{ZeroDispDB}}

The \object{ZeroDispDB} is a special case of the \object{UniformDB} for the Dirichlet
boundary conditions. The values in the database are the ones requested
by the Dirichlet boundary conditions, \filename{displacement-x}, \filename{displacement-y},
and \filename{displacement-z}, and are all set to zero. This makes it
trivial to set displacements to zero on a boundary. The examples discussed
in Chapter \vref{cha:examples} use this database.


\subsection{\object{SimpleGridDB} Spatial Database}

The \object{SimpleGridDB} object provides a much more efficient query
algorithm than \object{SimpleDB} in cases with a orthogonal grid. The
points do not need to be uniformly spaced along each coordinate
direction. Thus, in contrast to the \object{SimpleDB} there is an
implicit topology. Nevertheless, the points can be specified in any
order, as well as over a lower-dimension than the spatial dimension.
For example, one can specify a 2-D grid in 3-D space provided that the
2-D grid is aligned with one of the coordinate axes.

\object{SimpleGridDB} uses a simple ASCII file to specify the variation of
values (e.g., displacement field, slip field, physical properties) in
space. The file format is described in Section
\vref{sec:format:SimpleGridDB}.

As in the other Pyre objects, spatial database objects contain parameters
that can be set from the command line or using \filename{.cfg}
files. The parameters for a spatial database are:
\begin{inventory}
\propertyitem{label}{Label for the database, which is used in
  diagnostic messages.}
\propertyitem{query\_type}{Type of search query to perform. Values for this
parameter are ``linear'' and ``nearest'' (default).}
\propertyitem{filename}{Filename for the spatial database.}
\end{inventory}
An example of setting these parameters in a \filename{.cfg} file is:
\begin{cfg}
<p>label</p> = Material properties
<p>query_type</p> = linear
<p>filename</p> = mydb_grid.spatialdb
\end{cfg}

\subsection{SCEC CVM-H Spatial Database (\object{SCECCVMH})}
\label{sec:SCEC:CVM-H}

Although the \object{SimpleDB} implementation is able to specify arbitrarily
complex spatial variations, there are existing databases for physical
properties, and when they are available, it is desirable to access
these directly. One such database is the SCEC CVM-H database, which
provides seismic velocities and density information for much of southern
California. Spatialdata provides a direct interface to this database.
See Section \vref{sec:examples:twotet4-geoproj} for an example of
using the SCEC CVM-H database for physical properties of an elastic
material. The interface is known to work with versions 5.2 and 5.3
of the SCEC CVM-H. Setting a minimum wave speed can be used to replace
water and very soft soils that are incompressible or nearly incompressible
with stiffer, compressible materials. The Pyre properties for the
SCEC CVM-H are:
\begin{inventory}
\propertyitem{data\_dir}{Directory containing the SCEC CVM-H data
  files.}
\propertyitem{min\_vs}{Minimum shear wave speed. Corresponding minimum values
for the dilatational wave speed (Vp) and density are computed. Default
value is 500 m/s.}
\propertyitem{squash}{Squash topography/bathymetry to sea level (make the earth's
surface flat).}
\propertyitem{squash\_limit}{Elevation above which topography is squashed (geometry
below this elevation remains undistorted).}
\end{inventory}

Specify the physical properties of a linearly elastic, isotropic material
using the SCEC CVM-H in a \filename{.cfg} file.
\begin{cfg}
<h>[pylithapp.timedependent.materials.material]</h>
<p>db_properties</p> = spatialdata.spatialdb.SCECCVMH ; Set the database to the SCEC CVM-H

# Directory containing the database data files.
<p>db_properties.data_dir</p> = /home/johndoe/data/sceccvm-h/vx53

<p>db_properties.min_vs</p> = 500*m/s ; Default value
<p>db_properties.squash</p> = True ; Turn on squashing

# Only distort the geometry above z=-1km in flattening the earth
<p>db_properties.squash_limit</p> = -1000.0
\end{cfg}

\subsection{\object{CompositeDB} Spatial Database}

For some problems, a boundary condition or material property may have
subsets with different spatial variations. One example would be when
we have separate databases to describe the elastic and inelastic bulk
material properties for a region. In this case, it would be useful
to have two different spatial databases, e.g., a seismic velocity
model with Vp, Vs, and density values, and another database with the
inelastic physical properties. We can use the \object{CompositeDB}
spatial database for these cases. An example would be:
\begin{cfg}
<h>[pylithapp.timedependent.materials.maxwell]</h>
<p>label</p> = Maxwell material
<p>id</p> = 1
<f>db_properties</f> = spatialdata.spatialdb.CompositeDB
<f>db_properties.db_A</f> = spatialdata.spatialdb.SCECCVMH
<f>db_properties.db_B</f> = spatialdata.spatialdb.SimpleDB
<f>quadrature.cell</f> = pylith.feassemble.FIATSimplex
<p>quadrature.cell.dimension</p> = 3

<h>[pylithapp.timedependent.materials.maxwell.db_properties]</h>
<p>values_A</p> = [density, vs, vp]
<p>db_A.label</p> = Elastic properties from CVM-H
<p>db_A.data_dir</p> = /home/john/tools/vx53/bin
<p>db_A.squash</p> = False
<p>values_B</p> = [viscosity]
<p>db_B.label</p> = Vertically varying Maxwell material
<p>db_B.iohandler.filename<p> = ../spatialdb/mat_maxwell.spatialdb
\end{cfg}
Here we have specified a \object{CompositeDB} where the elastic properties
(density, vs, vp) are given by the SCEC
CVM-H, and viscosity is described by a \object{SimpleDB}
(\filename{mat\_maxwell.spatialdb}). The user must first
specify \facility{db\_properties} as a \object{CompositeDB}, and must
then give the two components of this database (in this case, \object{SCECCVMH} and
\object{SimpleDB}). The values to query in each of these databases
is also required. This is followed by the usual parameters for each
of the spatial databases. The \object{CompositeDB} provides a flexible
mechanism for specifying material properties or boundary conditions
where the variations come from two different sources.


\subsection{\object{TimeHistory} Database}

The \object{TimeHistory} database specifies the temporal variation in the
amplitude of a field associated with a boundary condition. It is used
in conjunction with spatial databases to provide spatial and temporal
variation of parameters for boundary conditions. The same time history
is applied to all of the locations, but the time history may be shifted
with a spatial variation in the onset time and scaled with a spatial
variation in the amplitude. The time history database uses a simple
ASCII file which is simpler than the one used by the \object{SimpleDB} spatial
database. The file format is described in Section \vref{sec:format:TimeHistoryIO}. 

As in the other Pyre objects, spatial database objects contain parameters
that can be set from the command line or using \filename{.cfg}
files. The parameters for a spatial database are:
\begin{inventory}
\propertyitem{label}{Label for the time history database, which is used in diagnostic
messages.}
\propertyitem{filename}{Filename for the time history database.}
\end{inventory}
An example of setting these parameters in a \filename{.cfg} file is:
\begin{cfg}
<p>label</p> = Displacement time history
<p>filename</p> = mytimehistory.timedb
\end{cfg}

% End of file
