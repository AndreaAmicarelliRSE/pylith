
\chapter{\label{cha:Extending}Extending PyLith}

One of the powerful features of using the Pyre framework in PyLith
is the ability to extend the functionality of the software without
altering any of the PyLith code. Any of the components can be replaced
with other compatible components. You are already familiar with this
feature from running the examples; when you set the spatial database
to UniformDB, SimpleDB, or SCECCVMH you are switching between different
compatible components for a spatial database facility. Modifying the
governing equations to include other physical processes requires changing
the data structures associated with the solution and altering the
PyLith code.

In this section we provide examples of how to extend PyLith for components
that users will most likely want to replace with their own custom
versions. You will need a familiarity with Python, Makefiles, and
C++ to write your own components. The primary steps in constructing
a component to extend PyLith's functionality include:
\begin{enumerate}
\item Setting up the source files for the component or set of components
based on the templates.
\item Edit the Python source file (\texttt{.py}) for the component.

\begin{enumerate}
\item Define the user-specified properties and facilities.
\item Transfer the user-specified data from the Python object to the corresponding
C++ object via calls to the SWIG interface object.
\end{enumerate}
\item Edit the C++ header (\texttt{.hh}) and implementation files (\texttt{.cc})
for the component.

\begin{enumerate}
\item Implement the methods required to satisfy the interface definition
of the component.
\item Implement the desired functionality of the component in C++.
\end{enumerate}
\item Edit the SWIG interface files (\texttt{.i}) that provide the glue
between Python and C++.
\item Edit the Python source file that tests the functionality of the component.
\item Run configure, build, install, and run the tests of the component.
\end{enumerate}

\section{\label{sec:Extending:SpatialDatabases}Spatial Databases}

PyLith provides several types of spatial databases that can be used
for specification of parameters associated with boundary conditions,
earthquake ruptures, and physical properties. In this example we demonstrate
how to provide a spatial database, UniformVelModel, for specifying
elastic properties. The source files are included in the source for
the spatialdata package in the \texttt{templates/spatialdb} directory.
The \texttt{README} file in \texttt{templates/spatialdb} provides
detailed instructions for the various steps involved, and the source
files contain numerous comments to help guide you through the customization
process. 

The UniformVelModel component provides uniform physical properties:
P-wave speed, S-wave speed, and density. Although this is a rather
trivial specification of physical properties that could be easily
done using a UniformDB, this example demonstrates how to create a
user-defined component that matches the requirements of a spatial
database for elastic physical properties. Adding additional physical
properties is simply a matter of including some additional values
in the spatial database. Furthermore, in cases where we are constructing
a spatial database for a seismic velocity model, the data points are
georeferenced. With our uniform physical properties we do not need
to worry about any differences in coordinate systems between our seismic
velocity model and points at which the model is queried. However,
in many cases we do, so we illustrate this functionality by using
a geographic projection as the coordinate system in our example.

Using a top-down approach, the first step is to determine what information
the user will need to supply to the component. Is the data for the
spatial database in a file or a series of files? If so, file names
and possible paths to a directory containing files with known names
might be necessary. Are there other parameters that control the behavior
of the component, such as a minimum shear wave speed? In our example
the user supplies values for the P-wave speed, S-wave speed, and density.
The user-supplied parameters become Pyre properties and facilities
in the Python source file. Because our user supplied parameters are
floating point values with dimensions, we create dimensional properties
\texttt{vs}, \texttt{vp}, and \texttt{density}. In addition to defining
the properties of the component, we also need to transfer these properties
to the C++ object that does the real work. This is done by calling
the C++ \texttt{vs()}, \texttt{vp()}, and \texttt{density()} accessor
functions that are accessible via the Python module created by SWIG.

In the C++ object we must implement the functions that are required
by the spatial database interface. These functions are listed near
the beginning of the UniformVelModel class definition at the top of
the C++ header file, \texttt{UniformVelModel.hh}. The C++ object also
includes the accessor functions that allow us to set the P-wave speed,
S-wave speed, and density values to the user-specified values in the
Python object. Additional information, such as a file name, parameters
defined as data structures, etc., would be set via similar accessor
functions. You can also add additional functions and data structures
to the C++ class to provide the necessary operations and functionality
of the spatial database. 

In SimpleDB we use a separate class to read in the spatial database
and yet another class to perform the actual query. In our example,
the C++ object also creates and stores the UTM zone 10 geographic
projection for the seismic velocity model. When the spatial database
gets a query for physical properties, we transform the coordinates
of the query point from its coordinate system to the coordinate system
of our seismic velocity model.

In order to use SWIG to create the Python module that allows us to
call C++ from Python, we use a ``main'' SWIG interface file (\texttt{\small{}spatialdbcontrib.i}
in this case) and then one for each object (\texttt{\small{}UniformVelModel.i}
in this case). This greatly simplifies keeping the Python module synchronized
with the C++ and Python code. The \texttt{\small{}UniformVelModel.i}
SWIG file is nearly identical to the corresponding C++ header file.
There are a few differences, as noted in the comments within the file.
Copying and pasting the C++ header file and then doing a little cleanup
is a very quick and easy way to construct a SWIG interface file for
a C++ object. Because very little changes from SWIG module to SWIG
module, it is usually easiest to construct the ``main'' SWIG interface
by copying and customizing an existing one.

Once the Python, C++, and SWIG interface files are complete, we are
ready to build the module. The \texttt{Makefile.am} file defines how
to compile and link the C++ library and generate the Python module
via SWIG. The \texttt{configure.ac} file contains the information
used to build a configure script. The configure script checks to make
sure it can find all of the tools needed to build the component (C++
compiler, Python, installed spatial database package, etc.). See the
README file for detailed instructions on how to generate the configure
script, and build and install the component.

We recommend constructing tests of the component to insure that it
is functioning properly before attempting to use it in an application.
The \texttt{\small{}tests} directory within \texttt{\small{}templates/spatialdb}
contains a Python script, \texttt{\small{}testcontrib.py}, that runs
the tests of the UniformVelModel component defined in \texttt{\small{}TestUniformVelModel.py}.
Normally, one would want to test each function individually to isolate
errors and create C++ tests as well as the Python tests included here.
In our rather simple example, we simply test the overall functionality
of the component. For examples of thorough testing, see the spatialdata
and PyLith source code.

Once you have built, installed, and tested the UniformVelModel, it
is time to use it in a simple example. Because the seismic velocity
model uses georeferenced coordinates, our example must also use georeferenced
coordinates. The dislocation example in the PyLith \texttt{examples/twocells/twotet4-geoproj}
directory uses UTM zone 11 coordinates. The spatial database package
will transform the coordinates between the two projections as defined
in the UniformVelModel \texttt{query()} function. The dislocation
example uses the SCEC CVM-H seismic velocity model. In order to replace
the SCEC CVM-H seismic velocity with our uniform velocity model, in
\texttt{pylithapp.cfg} we replace the lines
\begin{lyxcode}
db\_properties~=~spatialdata.spatialdb.SCECCVMH

db\_properties.data\_dir~=~/home/brad/data/sceccvm-h/vx53/bin
\end{lyxcode}
with the lines
\begin{lyxcode}
db\_properties~=~spatialdata.spatialdb.contrib.UniformVelModel
\end{lyxcode}
When you run the dislocation example, the \texttt{dislocation-statevars\_info.vtk}
file should reflect the use of physical properties from the uniform
seismic velocity with $\mu=1.69\times10^{10}\mathrm{Pa}$, $\lambda=1.6825\times10^{10}\mathrm{Pa}$,
and $\rho=2500\mathrm{kg/m^{3}}$.


\section{\label{sec:Extending:BulkConstitutiveModels}Bulk Constitutive Models}

PyLith includes several linearly elastic and inelastic bulk constitutive
models for 2D and 3D problems. In this example, we demonstrate how
to extend PyLith by adding your own bulk constitutive model. We reimplement
the 2D plane strain constitutive model while adding the current strain
and stress tensors as state variables. This constitutive model, \texttt{PlaneStrainState},
is not particularly useful, but it illustrates the basic steps involved
in creating a bulk constitutive model with state variables. The source
files are included with the main PyLith source code in the \texttt{templates/materials}
directory. The \texttt{README} file in \texttt{templates/materials}
provides detailed instructions for the various steps, and the source
files contain numerous comments to guide you through the customization
process.

In contrast to our previous example of creating a customized spatial
database, which involved gathering user-specified parameters via the
Pyre framework, there are no user-defined parameters for bulk constitutive
models. The specification of the physical properties and state variables
associated with the constitutive model is handled directly in the
C++ code. As a result, the Python object for the constitutive model
component is very simple, and customization is limited to simply changing
the names of objects and labels.

The properties and state variables used in the bulk constitutive model
are set using arguments to the constructor of the C++ \texttt{ElasticMaterial}
object. We define a number of constants at the top of the C++ file
and use the \texttt{Metadata} object to define the properties and
state variables. The C++ object for the bulk constitutive component
includes a number of functions that implement elasticity behavior
of a bulk material as well as several utility routines:
\begin{description}
\item [{\texttt{\_dbToProperties()}}] Computes the physical properties
used in the constitutive model equations from the physical properties
supplied in spatial databases.
\item [{\texttt{\_nondimProperties()}}] Nondimensionalizes the physical
properties used in the constitutive model equations.
\item [{\texttt{\_dimProperties()}}] Dimensionalizes the physical properties
used in the constitutive model equations.
\item [{\texttt{\_stableTimeStepImplicit()}}] Computes the stable time
step for implicit time stepping in quasi-static simulations given
the current state (strain, stress, and state variables).
\item [{\texttt{\_calcDensity()}}] Computes the density given the physical
properties and state variables. In most cases density is a physical
property used in the constitutive equations, so this is a trivial
function in those cases.
\item [{\texttt{\_calcStress()}}] Computes the stress tensor given the
physical properties, state variables, total strain, initial stress,
and initial strain.
\item [{\texttt{\_calcElasticConsts()}}] Computes the elastic constants
given the physical properties, state variables, total strain, initial
stress, and initial strain.
\item [{\texttt{\_updateStateVars()}}] Updates the state variables given
the physical properties, total strain, initial stress, and initial
strain. If a constitutive model does not use state variables, then
this routine is omitted.
\end{description}
Because it is sometimes convenient to supply physical properties for
a bulk constitutive model that are equivalent but different from the
ones that appear in the constitutive equations (e.g., P-wave speed,
S-wave speed, and density rather then Lame's constants $\mu,$$\lambda,$
and density), each bulk constitutive model component has routines
to convert the physical property parameters and state variables a
user specifies via spatial databases to the physical property properties
and state variables used in the constitutive model equations. In quasi-static
problems, PyLith computes an elastic solution for the first time step
($-\Delta t$ to $t$). This means that for inelastic materials, we
supply two sets of functions for the constitutive behavior: one set
for the initial elastic step and one set for the remainder of the
simulation. See the source code for the inelastic materials in PyLith
for an illustration of an efficient mechanism for doing this.

The SWIG interface files for a bulk constitutive component are set
up in the same manner as in the previous example of creating a customized
spatial database component. The ``main'' SWIG interface file (\texttt{materialscontrib.i}
in this case) sets up the Python module, and the SWIG interface file
for the component (\texttt{PlaneStrainState.i} in this case) defines
the functions that should be included in the Python module. Note that
because the C++ \texttt{ElasticMaterial} object defines a number of
pure virtual methods (which merely specify the interface for the functions
and do not implement default behavior), we must include many protected
functions in the SWIG interface file. If these are omitted, SWIG will
give a warning indicating that some of the functions remain abstract
(i.e., some pure virtual functions defined in the parent class \texttt{ElasticMaterial}
were not implemented in the child class \texttt{PlaneStrainState}),
and no constructor is created. When this happens, you cannot create
a \texttt{PlaneStrainState} Python object.

Once the Python, C++, and SWIG interface files are complete, you are
ready to configure and build the C++ library and Python module for
the component. Edit the \texttt{Makefile.am} file as necessary, then
generate the configure script, run \texttt{configure}, and then build
and install the library and module (see the \texttt{README} file for
detailed instructions).

Because most functionality of the bulk constitutive model component
is at the C++ level, properly constructed unit tests for the \texttt{component}
should include tests for both the C++ code and Python code. The C++
unit tests can be quite complex, and it is best to examine those used
for the bulk constitutive models included with PyLith. In this example
we create the Python unit tests to verify that we can create a \texttt{PlaneStrainState}
Python object and call some of the simple underlying C++ functions.
The source files are in the \texttt{templates/materials/tests} directory.
The \texttt{testcontrib.py} Python script runs the tests defined in
\texttt{TestPlaneStrainState.py}.

Once you have built, installed, and tested the \texttt{PlaneStrainState}
component, it is time to use it in a simple example. You can try using
it in any of the 2D examples. For the examples in \texttt{examples/twocells/twoquad4},
in \texttt{pylithapp.cfg} replace the line
\begin{lyxcode}
material~=~pylith.materials.ElasticPlaneStrain
\end{lyxcode}
with the line
\begin{lyxcode}
material~=~pylith.materials.contrib.PlaneStrainState
\end{lyxcode}
or simply add the command line argument
\begin{lyxcode}
-{}-timedependent.homogeneous.material=pylith.materials.contrib.PlaneStrainState
\end{lyxcode}
when running any of the 2D examples. The output should remain identical,
but you should see the \texttt{PlaneStrainState} object listed in
the information written to \texttt{stdout}.


\section{\label{sec:Extending:FaultConstitutiveModels}Fault Constitutive
Models}

PyLith includes two of the most widely used fault constitutive models,
but there are a wide range of models that have been proposed to explain
earthquake source processes. In this example, we demonstrate how to
extend PyLith by adding your own fault constitutive model. We implement
a linear viscous fault constitutive model wherein the perturbation
in the coefficient of friction is linearly proportional to the slip
rate. This constitutive model, \texttt{ViscousFriction}, is not particularly
useful, but it illustrates the basic steps involved in creating a
fault constitutive model. The source files are included with the main
PyLith source code in the \texttt{templates/friction} directory. The
\texttt{README} file in \texttt{templates/friction} provides detailed
instructions for the various steps, and the source files contain numerous
comments to guide you through the customization process.

Similar to our previous example of creating a customized bulk constitutive
model, the parameters are defined in the C++ code, not in the Pyre
framework. As a result, the Python object for the fault constitutive
model component is very simple and customization is limited to simply
changing the names of objects and labels.

The properties and state variables used in the fault constitutive
model are set using arguments to the constructor of the C++ \texttt{FrictionModel}
object, analogous to the \texttt{ElasticMaterial} object for bulk
constitutive models. In fact, both types of constitutive models used
the same underlying C++ object (\texttt{Metadata::ParamDescription})
to store the description of the parameters and state variables. We
define a number of constants at the top of the C++ file and use the
\texttt{Metadata} object to define the properties and state variables.
The C++ object for the fault constitutive component includes a number
of functions that implement friction as well as several utility routines:
\begin{description}
\item [{\texttt{\_dbToProperties()}}] Computes the physical properties
used in the constitutive model equations from the physical properties
supplied in spatial databases.
\item [{\texttt{\_nondimProperties()}}] Nondimensionalizes the physical
properties used in the constitutive model equations.
\item [{\texttt{\_dimProperties()}}] Dimensionalizes the physical properties
used in the constitutive model equations.
\item [{\texttt{\_dbToStateVars()}}] Computes the initial state variables
used in the constitutive model equations from the initial values supplied
in spatial databases.
\item [{\texttt{\_nondimStateVars()}}] Nondimensionalizes the state variables
used in the constitutive model equations.
\item [{\texttt{\_dimStateVars()}}] Dimensionalizes the state variables
used in the constitutive model equations.
\item [{\texttt{\_calcFriction()}}] Computes the friction stress given
the physical properties, state variables, slip, slip rate, and normal
traction.
\item [{\texttt{\_updateStateVars()}}] Updates the state variables given
the physical properties, slip, slip rate, and normal traction.
\end{description}
If a constitutive model does not use state variables, then the state
variable routines are omitted. 

Because it is sometimes convenient to supply physical properties for
a fault constitutive model that are equivalent but different from
the ones that appear in the constitutive equations, each fault constitutive
model component has routines to convert the physical property parameters
and state variables a user specifies via spatial databases to the
physical property properties and state variables used in the constitutive
model equations. 

The SWIG interface files for a fault constitutive component are set
up in the same manner as in the previous examples of creating a bulk
constitutive model or a customized spatial database component. The
``main'' SWIG interface file (\texttt{frictioncontrib.i} in this case)
sets up the Python module, and the SWIG interface file for the component
(\texttt{ViscousFriction.i} in this case) defines the functions that
should be included in the Python module. Note that because the C++
\texttt{FrictionModel} object defines a number of pure virtual methods
(which merely specify the interface for the functions and do not implement
default behavior), we must include many protected functions in the
SWIG interface file. If these are omitted, SWIG will give a warning
indicating that some of the functions remain abstract (i.e., some
pure virtual functions defined in the parent class \texttt{FrictionModel}
were not implemented in the child class \texttt{ViscousFriction}),
and no constructor is created. When this happens, you cannot create
a \texttt{ViscousFriction} Python object.

Once the Python, C++, and SWIG interface files are complete, you are
ready to configure and build the C++ library and Python module for
the component. Edit the \texttt{Makefile.am} file as necessary, then
generate the configure script, run \texttt{configure}, and then build
and install the library and module (see the \texttt{README} file for
detailed instructions).

Because most functionality of the fault constitutive model component
is at the C++ level, properly constructed unit tests for the \texttt{component}
should include tests for both the C++ code and Python code. The C++
unit tests can be quite complex, and it is best to examine those used
for the fault constitutive models included with PyLith. In this example
we create the Python unit tests to verify that we can create a \texttt{ViscousFriction}
Python object and call some of the simple underlying C++ functions.
The source files are in the \texttt{templates/friction/tests} directory.
The \texttt{testcontrib.py} Python script runs the tests defined in
\texttt{TestViscousFriction.py}.

Once you have built, installed, and tested the \texttt{ViscousFriction}
component, it is time to use it in a simple example. You can try using
it in any of the 2D or 3D examples. For the examples in \texttt{examples/bar\_shearwave/quad4,}
in \texttt{shearwave\_staticfriction.cfg} replace the line
\begin{lyxcode}
friction~=~pylith.friction.StaticFriction
\end{lyxcode}
with the line
\begin{lyxcode}
friction~=~pylith.friction.contrib.ViscousFriction
\end{lyxcode}
or simply add the command line argument
\begin{lyxcode}
-{}-timedependent.interfaces.fault.friction=pylith.friction.contrib.ViscousFriction
\end{lyxcode}
when running any of the friction examples. You will also need to supply
a corresponding spatial database with the physical properties for
the viscous friction constitutive model.
