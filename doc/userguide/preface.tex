
\chapter{Preface}


\section{About This Document}

This document is organized into two parts. The first part begins with
an introduction to PyLith and discusses the types of problems that
PyLith can solve and how to run the software; the second part provides
appendices and references.


\section{Who Will Use This Documentation}

This documentation is aimed at two categories of users: scientists
who prefer to use prepackaged and specialized analysis tools, and
experienced computational Earth scientists. Of the latter, there are
likely to be two classes of users: those who just run models, and
those who modify the source code. Users who modify the source are
likely to have familiarity with scripting, software installation,
and programming, but are not necessarily professional programmers.

\section{Conventions}

\warning{This is a warning.}
\important{This is something important.}
\tip{This is a tip, helpful hint, or suggestion.}

For features recently added to PyLith, we show the version number when
they were added.\newfeature{\pylithVersion}

\subsection{Command Line Arguments}

Exmaple of a command line argument: \commandline{-{}-help}.

\subsection{Filenames and Directories}

Example of filenames and directories: \filename{pylith}, \filename{/usr/local}.

\subsection{Unix Shell Commands}

Commands entered into a Unix shell (i.e., terminal) are shown in a
box. Comments are delimited by the \# character. We use 
{\tt \$} to indicate the bash shell prompt.
\begin{shell}
# This is a comment.
$ ls -l
\end{shell}

\subsection{Excerpts of cfg Files}

Example of an excerpt from a \filename{.cfg} file:
\begin{cfg}
# This is a comment.
<h>[pylithapp.problem]</h>
<p>timestep</p> = 2.0*s ; Time step comment.
<f>bc</f> = [x_pos, x_neg]
\end{cfg}

\section{Citation}

The Computational Infrastructure for Geodynamics (CIG) (\url{geodynamics.org})
is making this source code available to you at no cost in hopes that
the software will enhance your research in geophysics. A number of
individuals have contributed a significant portion of their careers
toward the development of this software. It is essential that you
recognize these individuals in the normal scientific practice by citing
the appropriate peer-reviewed papers and making appropriate acknowledgments
in talks and publications. The preferred way to generate the list
of publications (in Bib\TeX{} format) to cite is to run your simulations
with the \commandline{-{}-include-citations} command line argument, or
equivalently, the \commandline{-{}-petsc.citations} command line argument.
The \commandline{-{}-help-citations} command line argument will generate
the Bib\TeX{} entries for the references mentioned below.

The following peer-reviewed paper discussed the development of PyLith:
\begin{itemize}
\item Aagaard, B. T., M. G. Knepley, and C. A. Williams (2013). A
  domain decomposition approach to implementing fault slip in
  finite-element models of quasi-static and dynamic crustal
  deformation, \textit{Journal of Geophysical Research: Solid Earth},
  118, doi: 10.1002/jgrb.50217.
\end{itemize}
To cite the software and manual, use:
\begin{itemize}
\item Aagaard, B., M. Knepley, C. Williams (2017), \emph{PyLith
  \pylithVersion.} Davis, CA: Computational Infrastructure of
  Geodynamics. DOI: \pylithDOI.
\item Aagaard, B., M. Knepley, C. Williams (2017), \emph{PyLith User
  Manual, Version \pylithVersionNumber.} Davis, CA: Computational
  Infrastructure of Geodynamics. URL:
  geodynamics.org/cig/software/github/pylith/\pylithVersion/pylith-\pylithVersionNumber\_manual.pdf
\end{itemize}

\section{Support}

Current PyLith development is supported by the CIG, and internal GNS
Science \url{www.gns.cri.nz} and U.S. Geological Survey \url{www.usgs.gov}
funding. Pyre development was funded by the Department of Energy's
\url{www.doe.gov/engine/content.do} Advanced Simulation and Computing
program and the National Science Foundation's Information Technology
Research (ITR) program.

This material is based upon work supported by the National Science
Foundation under Grants No. 0313238, 0745391, 1150901, and
EAR-1550901. Any opinions, findings, and conclusions or
recommendations expressed in this material are those of the author(s)
and do not necessarily reflect the views of the National Science
Foundation.


\section{Acknowledgments}

Many members of the community contribute to PyLith through reporting
bugs, suggesting new features and improvements, running benchmarks,
and asking questions about the software. In particular, we thank Surendra
Somala for contributing to the development of the fault friction implementation.


\section{Request for Comments}

Your suggestions and corrections can only improve this documentation.
Please report any errors, inaccuracies, or typos to the CIG Short-Term
Tectonics email list \url{cig-short@geodynamics.org} or create a
GitHub pull request.
