\documentclass{article}[10pt]
\usepackage{amsmath}

\renewcommand{\matrix}[1]{\bar{#1}}
\newcommand{\Adiag}{\ensuremath{A_\mathit{diag}}}

% ------------------------------------------------------------------
% Basic page layout
\setlength{\textheight}{9.0in}
\setlength{\textwidth}{7.0in}
\setlength{\topmargin}{-20pt}
\setlength{\headheight}{16pt}
\setlength{\headsep}{4pt}
\setlength{\oddsidemargin}{-0.25in}
\setlength{\footskip}{24pt}
%
\setlength{\floatsep}{12pt}
\setlength{\textfloatsep}{12pt}
\setlength{\intextsep}{12pt}
\setlength{\abovecaptionskip}{0pt}
\setlength{\belowcaptionskip}{6pt}
%
\setlength{\parindent}{0.25in}
\setlength{\parskip}{6pt}
%
% ------------------------------------------------------------------
% Font Sizes
\usepackage{times}
%
% ======================================================================
\begin{document}

% ----------------------------------------------------------------------
\section{Schur Complement Preconditioner}

We have a Jacobian of the form
\begin{equation}
  J = \left( \begin{array}{cc}
    A & C^T \\
    C & 0
  \end{array} \right).
\end{equation}
We use the Schur complement of block A to examine the form of $J^{-1}$,
\begin{equation}
  J^{-1} = \left( \begin{array}{cc}
    A^{-1}+A^{-1} C^{T}(-C A^{-1} C^{T})^{-1} C A^{-1} & 
    -A^{-1}C^{T}(-C A^{-1} C^{T})^{-1} \\
    -(-C A^{-1} C^{T})^{-1} C A^{-1} & -(C A^{-1} C^T)^{-1}
  \end{array} \right),
\end{equation}
A suitable block diagonal $P^{-1}$ is
\begin{equation}
  P^{-1} = \left( \begin{array}{cc}
    A^{-1} & 0 \\
    0 & -(C A^{-1} C^T)^{-1}
  \end{array} \right),
\end{equation}
which leads to
\begin{equation}
  P = \left( \begin{array}{cc}
    A & 0 \\
    0 & C A^{-1} C^T
  \end{array} \right).
\end{equation}

We provide PETSc with $P$ so that it can create $P^{-1}$. Using the
field split preconditioner, we form
\begin{equation}
  P = \left( \begin{array}{cc}
    A_\mathit{ml} & 0 \\
    0 & P_f
  \end{array} \right),
\end{equation}
where we use the ML package to form $A_\mathit{ml}$ and we create a
custom matrix for the portion of the preconditioner associated with
the Lagrange constraints, $P_f$. Let $n$ be the number of conventional
degrees of freedom and $l$ be the number of Lagrange constraints. This
means $A$ and $P$ are $(n+l) \times (n+l)$, $A$ and $A_\mathit{ml}$
are $n \times n$, $C$ is $l \times n$, and $P_f$ is $l \times l$.
We let $P_f$ be the the diagonal approximation of $C A^{-1} C^T$,
\begin{equation}
  P_f = \text{diagonal}(C A_\mathit{diag}^{-1} C^T).
\end{equation}
Using the {\tt multiplicative} field split type, PETSc will form
$P^{-1}$ as
\begin{equation}
  P^{-1} = \left( \begin{array}{cc}
    A_\mathit{ml}^{-1} & -A_\mathit{ml}^{-1} C^T ?? \\
    0 & P_f^{-1}
  \end{array} \right).
\end{equation}



% ======================================================================
\end{document}